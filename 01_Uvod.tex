\chapter*{Introduction}
\addcontentsline{toc}{chapter}{Introduction}

The group of circular units is an important object of study in modern number theory. In contrast to the case of cyclotomic fields, there are several possible definitions of a group of circular units of a general number field. One of the most well known of these definitions is due to Sinnott and appeared in the paper \citep{SinnottAb}, where he generalized the results from \citep{SinnottCirc}. Sinnott's circular units generalize many previous groups of units due to Kummer, Hasse, Leopoldt, Gras, Gillard and others, and have proven to have many interesting applications; for example they have deep connections to class groups, appear in Iwasawa theory and form an example of an Euler system. In order to do calculations, sometimes it's useful to have an explicit $\Z$-basis of this group at hand. However, such a basis is known only in a few special cases, for example for cyclotomic fields, composita of quadratic fields, abelian fields ramified at at most two primes or certain abelian fields ramified at three primes (see \citep{Kucera1992}, \citep{Kucera1996}, \citep{Dohmae1996}, \citep{Dohmae1997} and \citep{Kucera2016}).
\bigskip

The goal of this thesis is to construct explicit $\Z$-bases of the groups of circular numbers and circular units (in Sinnott's sense) in yet another case, namely that of a real abelian field ramified at exactly four primes and satisfying some additional conditions, as well as to study the Ennola relations that occur along the way, following the approach in \citep{Kucera2016}. \medskip

The thesis is subdivided into four chapters. In the first of them, we recall the basic definitions and results about abelian fields and the groups of circular numbers and units, while in the second, we describe more precisely the class of fields we will study and we lay foundations for the work in the rest of the thesis.\medskip

The third chapter forms the core of the thesis. We describe here the general strategy we will use to find the $\Z$-bases, and then we do so in five different families of cases of increasing difficulty. Finally in the short fourth chapter, we explain how these constructions relate to the module of relations and discuss some partial results about the relevant Ennola relations.\medskip

The reader is assumed to know the basics of Galois theory, module theory and algebraic number theory, especially ramification theory. The knowledge of the theory of Dirichlet characters in the scope of \citep{washington1997}, Chapter 3 is also welcome, although not strictly required. During Chapter \ref{bases} of the thesis, the reader is also strongly advised to draw pictures to get a better grasp of the exposition. \medskip

This thesis was created in the typographic system \LaTeX. For some of the computations, the computer algebra system SAGE was used.
%Tato práce byla vysázena v systému \LaTeX, při něktěrých výpočtech byl použit systém SAGE.
%%%%%%%%%%%%%%%%%%%%%%%%%%%%%%%%%%%%
%%%%%%%%% GENERUJ TEXT %%%%%%%%%%%%%

%\shorthandoff{-}\lipsum[50-60]\shorthandon{-}

%%%%%%%%%%%%%%%%%%%%%%%%%%%%%%%%%%%%


