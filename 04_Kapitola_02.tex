\chapter{The construction of bases of circular numbers and circular units}\label{bases}
%\texorpdfstring{$D^+$}{D^+} and \texorpdfstring{$C^+$}{C^+}}

\section{General strategy}
Our goal will be to find explicit $\Z$-bases of $D^+$ and $C^+$.%(it can then be easily modified in order to obtain a $\Z$-basis of $C^+$). 
 To achieve this, we will build upon the results in \citep{Kucera2016}. The generators of $D^+$ are subject to norm relations that correspond to the sum of all elements of the respective inertia groups $T_i$. Namely, let $$R_i=\sum_{u=0}^{a_i-1}\sigma_i^u,\, N_i=\sum_{u=0}^{n_i-1}\sigma_i^{ua_i}.$$ 
Then the norm operator from $K$ to $K_jK_lK_h$ can be given as $R_iN_i$, because both are equal to the sum of all elements from $T_i$. Moreover, Lemma \ref{genus} implies that $$\Gal(k/k\cap K_jK_lK_h)\cong \Gal(K/ K_jK_lK_h)=T_i,$$
where the first isomorphism is given by restriction, hence $R_iN_i$ also acts as the norm operator from $k$ to $k\cap K_jK_lK_h$.
%Then the norm operators from $k$ to a maximal subfield ramified at three primes can be given as $R_iN_i$ (i.e., the sum of all elements of $T_i$). 
If we denote the congruence corresponding to the canonical projection $\Z[G]\to \Z[G/H]$ by $\equiv$, then we have (using Lemma \ref{tau}) $$N_4\equiv \sum_{u=0}^{n_4-1}\sigma_1^{ua_1}\sigma_2^{ua_2}\sigma_3^{ua_3}.$$ Note that any subgroup of $k^{\times}$ is naturally a $\Z[G/H]$-module, since the action of $H$ on $k$ is trivial.

Moreover, we will denote the congruence corresponding to the composition of canonical projections $$\Z[G]\to \Z[G/H]\to \Z[G/H]/(R_1N_1,R_2N_2,R_3N_3,R_4N_4)$$ by $\sim$, where $(R_1N_1,R_2N_2,R_3N_3,R_4N_4)$ is the ideal generated in $\Z[G/H]$ by the images of the elements $R_iN_i$. Lemma \ref{units} shows that $\eta\in C^+$ , therefore 
%is a unit, so 
by Lemma \ref{NoFrob}, we have $\rho R_iN_i\cdot \eta\in C^+(k\cap K_jK_lK_h)$ for any $\rho\in G$. %When we apply any element of this ideal to the highest generator $\eta$, we will obtain a multiplicative $\Z$-linear combination of circular units belonging to a subfield $k\cap K_iK_jK_l$ with less ramified primes. 
We will make use of this extensively, because explicit $\Z$-bases of $D^+(k\cap K_iK_jK_l)$ and $C^+(k\cap K_iK_jK_l)$ have already been constructed in~\citep{Kucera2016}, as the following lemma shows.
 %is already known thanks to the following lemma, which shows that the fields $$k\cap K_1K_2K_3,k\cap K_1K_2K_4,k\cap K_1K_3K_4,k\cap K_2K_3K_4$$ satisfy the assumptions of \citep{Kucera2016}.

\begin{lemma}\label{Azar}
%The fields $$k\cap K_1K_2K_3,k\cap K_1K_2K_4,k\cap K_1K_3K_4,k\cap K_2K_3K_4$$ satisfy the assumptions of \citep{Kucera2016}. More explicitly, 
The field $k\cap K_iK_jK_l$ satisfies the assumptions of \citep{Kucera2016}. In other words, if $K'$ is the genus field of $k\cap K_iK_jK_l$, then $\Gal(K'/k\cap K_iK_jK_l)$ is cyclic and the inertia subgroups of $\Gal(K'/\Q)$ are all cyclic.
\end{lemma}
\begin{proof}
%It's clear that these fields are all real, abelian (their absolute Galois groups are quotients of $G$) and ramified at three primes. By symmetry, it suffices to prove the rest of the statement only for the field $k':=k\cap K_1K_2K_3$.

It's clear that $k\cap K_iK_jK_l$ is real, abelian (its absolute Galois group is a quotient of $G$) and ramified at three primes. By symmetry between the ramified primes, we can take $\{i,j,l\}=\{1,2,3\}$ in the rest of the proof and we will denote $k':=k\cap K_1K_2K_3$ to improve readability.
%it suffices to prove the rest of the statement only for the field $k':=k\cap K_1K_2K_3$.

Now let $K'$ be the genus field of $k'$, and for any $u\in\{1,2,3\}$, let $K'_u$ be the maximal subfield of $K'$ ramified only at $p_u$ and $T_u'$ be the inertia subgroup of $\Gal(K'/\Q)$ corresponding to $p_u$.
%Now fix $i,j,l$ (where $\{i,j,l,h \}=\{1,2,3,4\}$, as usual) and let $k'=k\cap K_iK_jK_l$, $K'$ be the genus field of $k'$, $K'_i$ be the maximal subfield of $K'$ ramified only at $p_i$ and $T_i'$ be the inertia subgroup of $\Gal(K'/\Q)$ corresponding to $p_i$. 
Then by Lemma \ref{genus}, we have $K_u'\subseteq K_u$ (using the alternate characterization of $K_u$), hence $T'_i\cong \Gal(K_u'/\Q)$ is isomorphic to a quotient of the cyclic group $\Gal(K/\Q)\cong T_i$, so it must also be cyclic.

Finally note that by Lemma \ref{genus}, we have $K'=K_1'K_2'K_3'\subseteq K_1K_2K_3$ and $kK_1K_2K_3=K$, hence $\Gal(K'/k')=\Gal(K_1'K_2'K_3'/k\cap K_1K_2K_3)$ is a quotient of $$\Gal(K_1K_2K_3/k\cap K_1K_2K_3)\cong \Gal(K/k),$$
which is cyclic. This concludes the proof.

\begin{center}
\begin{tikzpicture}
  \node (a) at (0,2)  {$K=kK_1K_2K_3$};
  \node (c) at (-2,-1)  {$k$};
  \node (d) at (2,0)  {$K_1K_2K_3$};
  \node (g) at (2,-2)  {$K'=K_1'K_2'K_3'$};
  \node (e) at (0,-4)  {$k'=k\cap K_1K_2K_3$};
  \node (f) at (0,-6)  {$\Q$};
\draw   (a) -- (c) -- (e) -- (g) -- (d) -- (a)
  (e) -- (f);
%  \draw   (a) -- node [above left]{} (c) -- (e) -- (g) -- node [below right ]{} (d) -- node [above right]{} (a)
%  (e) -- node [left]{}{} (f);
\end{tikzpicture}
\end{center}

\end{proof}

Using the results in \citep{Kucera2016}, we can thus take the $\Z$-bases of $$D^+(k\cap K_1K_2K_3),D^+(k\cap K_1K_2K_4),D^+(k\cap K_1K_3K_4),D^+(k\cap K_2K_3K_4)$$
and we will denote their union by $B_D$. Analogously, we can take the $\Z$-bases of $$C^+(k\cap K_1K_2K_3),C^+(k\cap K_1K_2K_4),C^+(k\cap K_1K_3K_4),C^+(k\cap K_2K_3K_4)$$
and denote their union by $B_C$. Note that $B_D$ and $B_C$ contain the same conjugates of $\eta_I$ for each $I\subsetneq \{1,2,3,4\}$, $|I|\geq 2$.
%$B_C\subseteq B_D$ and $B_D\setminus B_C$ contains only some conjugates of $\eta_{\{1\}},\eta_{\{2\}},\eta_{\{3\}},\eta_{\{4\}}.$
%Lemma \ref{F} now implies that $\eta^{R_iN_i}\in C^{+}$, so it can be expressed as a multiplicative combination of the elements of $B_C$

To construct a $\Z$-basis of $D^+$ (or $C^+$), we will take the union of $B_D$ (or $B_C$, respectively) with a set $B$ of suitably chosen conjugates of the highest generator $\eta$. In order to have a chance to obtain a $\Z$-basis of $D^+$, this set should have cardinality
\begin{equation}\label{Ndef}
\begin{split}
N:&=[k:\Q]+4-1-|B_D|\\
&=[k:\Q]+3-\sum_{i,j,l}([k\cap K_iK_jK_l:\Q]+2)+\sum_{i,j}([k\cap K_iK_j:\Q]+1)-\sum_{i}[k\cap K_i:\Q]\\
&=a_1a_2a_3a_4\frac{m^3}{r_1r_2r_3r_4}-\sum_{i,j,l}a_ia_ja_l\frac{m^2}{r_ir_jr_l}+\sum_{i,j}
a_ia_js_{ij}\frac{m}{r_ir_j}-\sum_{i}a_i+1
\end{split}
\end{equation}
by Proposition \ref{Drank} and using the principle of inclusion and exclusion (due to the fact that these bases were constructed \enquote{inductively}). Note that all conjugates of $\eta$ are units by Lemma \ref{units}, so this number $N$ will remain the same in the case of constructing a $\Z$-basis of $C^+$. Thus we do not have to distinguish between the cases of $D^+$ and $C^+$ anymore and we can take the set $B$ to be the same for both of them.

We cannot guarantee at the moment that the union of all these conjugates is not linearly dependent, but if we will show how to obtain all the missing conjugates of $\eta$ using the relations $$R_1N_1\sim 0, R_2N_2\sim 0, R_3N_3\sim 0, R_4\sum_{u=0}^{n_4-1}\sigma_1^{ua_1}\sigma_2^{ua_2}\sigma_3^{ua_3}\sim 0$$
and their $\Z[G]$-linear combinations, it will follow that we really have a $\Z$-basis thanks to the discussion just above Lemma \ref{Azar}. A typical way to do that will be the following: if $R\sim 0$ for some $R\in\Z[G]$ and $R\cdot \eta$ is a product of conjugates of $\eta$ such that we can already generate all of them except for precisely one, then we can generate the last one as well, because $R\cdot \eta$ can also be expressed as a $\Z$-linear combination of elements in $B_C$.

We will always refer to the conjugates of $\eta$ by their coordinates $x_1,x_2,x_3,x_4$ according to Proposition \ref{gal}. This allows us to visualise $\Gal(k/\Q)$ geometrically as a discrete (at most) four-dimensional cuboid.

\section{The case $r_1=r_2=r_3=r_4=1$}\label{bases1}
In this case, we have
$$\Gal(k/\Qbb)\cong
 \{\restr{\sigma_1^{x_1}\sigma_2^{x_2}\sigma_3^{x_3}\sigma_4^{x_4}}{k};~  0\leq x_1<a_1m, 0\leq x_2<a_2m, 0\leq x_3<a_3m,0\leq x_4<a_4\},$$
 $$s_{12}=s_{13}=s_{14}=s_{23}=s_{24}=s_{34}=1,$$
$$R_1N_1\sim 0, R_2N_2\sim 0, R_3N_3 \sim 0, R_4 \sum_{u=0}^{m-1}\sigma_1^{a_1u}\sigma_2^{a_2u}\sigma_3^{a_3u} \sim 0$$
and
\begin{align*}
N&=a_1a_2a_3a_4m^3-(a_1a_2a_3+a_1a_2a_4+a_1a_3a_4+a_2a_3a_4)m^2\\
&\hp+(a_1a_2+a_1a_3+a_1a_4+a_2a_3+a_2a_4+a_3a_4)m-a_1-a_2-a_3-a_4+1.\\
&=(a_1m-1)(a_2m-1)(a_3m-1)(a_4-1)+(a_1m-1)(a_2m-1)(a_3m-1)\\
&\hp-a_1a_2a_3m^2+(a_1a_2+a_1a_3+a_2a_3)m-a_1-a_2-a_3+1\\
&=(a_1m-1)(a_2m-1)(a_3m-1)(a_4-1)+(a_1m-1)(a_2(m-1)-1)(a_3m-1)\\
&\hp+(a_1-1)(a_3m-1)+a_3(m-1)
\end{align*}
We will define $B_1$ as the set of the following $N$ conjugates $\eta^{\sigma_1^{x_1}\sigma_2^{x_2}\sigma_3^{x_3}\sigma_4^{x_4}}$:

\begin{itemize}
\item $0\le x_1 <a_1m-1$, $0\le x_2 < a_2m-1$, $ 0 \le x_3<a_3m-1$, $ 1\le x_4<a_4$,
\item $0 \le x_1 <a_1m-1$, $0\le x_2 < a_2(m-1)-1$, $0 \le x_3<a_3m-1$, $ x_4=0$,
\item $0\le x_1 < a_1-1$, $ x_2 =a_2(m-1)-1$, $ 0\leq x_3 < a_3m-1$, $ x_4=0$,
\item $x_1=a_1-1$, $ x_2 =a_2(m-1)-1$, $ 0\leq x_3 < a_3(m-1)$, $ x_4=0$.
\end{itemize}

First we will recover the cases $0<x_4<a_4$, $x_1=a_1m-1$ or $x_2=a_2m-1$ or $x_3=n_3-1$ using the relations $R_1N_1\sim 0, R_2N_2\sim 0, R_3N_3\sim 0$. From now on, we only need to deal with the cases where $x_4=0$.
\paragraph*{}
Next, we will recover the cases $$x_1=a_1m-1, 0\le x_2 < a_2(m-1)-1, 0\leq x_3 <a_3m-1$$ using the relation $R_1N_1\sim 0$ and subsequently the cases $$0\leq x_1 < a_1m,  0\le x_2 < a_2(m-1)-1, x_3 = a_3m-1$$ and $$0\le x_1 < a_1-1, x_2 = a_2(m-1)-1, x_3 = a_3m-1$$ using the relation $R_3N_3\sim 0$.

\paragraph*{}
At this moment, we are only missing the conjugates $\eta^{\sigma_1^{x_1}\sigma_2^{x_2}\sigma_3^{x_3}}$ with $$a_1\le x_1< a_1m, x_2=a_2(m-1)-1, 0\le x_3<a_3m,$$ $$0\le x_1< a_1m, a_2(m-1)\le x_2<a_2m, 0\le x_3<a_3m$$ and $$x_1=a_1-1, x_2=a_2(m-1)-1, a_3(m-1)\le x_3<a_3m.$$ To continue, we need to define an auxiliary relation.

Let 
\begin{equation}\label{Gamma1def}
\Gamma:=\sigma_2^{a_2(m-2)}-\sum_{u=0}^{m-3}\sum_{v=1}^{u+1}\sigma_1^{a_1v}\sigma_2^{a_2u}\in\Z[G].
\end{equation}

\begin{lemma}\label{Gamma1}
We have $$R_1R_2R_4\Gamma\sum_{u=0}^{m-1}\sigma_3^{a_3u}\sim 0.$$
\end{lemma}
\begin{proof}
We have 
\begin{align*}
\sigma_1^{a_1}\sigma_2^{a_2}\Gamma &=\sigma_1^{a_1}\sigma_2^{a_2(m-1)}-\sum_{u=1}^{m-2}\sum_{v=2}^{u+1}\sigma_1^{a_1v}\sigma_2^{a_2u}\\
&=\sigma_1^{a_1}N_2-\sigma_1^{a_1}\sum_{u=0}^{m-2}\sigma_2^{a_2u}-\sum_{u=0}^{m-2}\sum_{v=2}^{u+1}\sigma_1^{a_1v}\sigma_2^{a_2u}\\
&=\sigma_1^{a_1}N_2-\sum_{u=0}^{m-2}\sum_{v=1}^{u+1}\sigma_1^{a_1v}\sigma_2^{a_2u}\\
&=\sigma_1^{a_1}N_2-\sigma_2^{a_2(m-2)}+\Gamma-\sum_{v=1}^{m-1}\sigma_1^{a_1v}\sigma_2^{a_2(m-2)}\\
&=\sigma_1^{a_1}N_2-\sigma_2^{a_2(m-2)}N_1+\Gamma,
\end{align*}
which implies $$\sigma_1^{a_1}\sigma_2^{a_2}R_1R_2\Gamma \sim R_1R_2\Gamma.$$
Using this repeatedly, we obtain
\begin{align*}
R_1R_2R_4\Gamma\sum_{u=0}^{m-1}\sigma_3^{a_3u} \sim R_1R_2\Gamma \left(R_4 \sum_{u=0}^{m-1}\sigma_1^{a_1u}\sigma_2^{a_2u}\sigma_3^{a_3u}\right)\sim 0,
\end{align*}
as needed.
\end{proof}

Thanks to Lemma \ref{Gamma1}, we will recover all the cases $$x_1=a_1-1, x_2=a_2(m-1)-1, a_3(m-1)\le x_3<a_3m,x_4=0$$ using the relation $\sigma_3^{w}R_1R_2R_4\Gamma\sum_{u=0}^{m-1}\sigma_3^{a_3u}\sim 0$ for all $0\le w< a_3$, since we can already recover the conjugates of $\eta$ corresponding to the summands in this relation arising from the double sum in \eqref{Gamma1def} (their exponents of $\sigma_2$ are between $0$ and $a_2(m-2)-1$) as well as the conjugates of $\eta$ corresponding to the summands arising from the first summand in \eqref{Gamma1def}, except for precisely one.

Next, we will recover all the cases $$0\leq x_1 < a_1m, a_2(m-1)\leq x_2 < a_2m-1, 0\leq x_3 <a_3m,x_4=0$$
using the relation $R_4\sum_{u=0}^{m-1}\sigma_1^{a_1u}\sigma_2^{a_2u}\sigma_3^{a_3u}\sim 0$, due to the fact that for any two different conjugates of $\eta$ used in this relation, the difference of their exponents of $\sigma_2$ is divisible by $a_2$ (and we have already recovered all of them except exactly one).
 After this, we can recover the cases $$0\leq x_1 < a_1, x_2 = a_2m-1, 0\leq x_3 <a_3m,x_4=0$$ using the relation $R_2N_2\sim 0$.
\paragraph*{}
Finally, we will use induction with respect to $v=0,1,\dots, m-1$ to show that we can recover the conjugates $\eta^{\sigma_1^{x_1}\sigma_2^{x_2}\sigma_3^{x_3}}$ with $$va_1\leq x_1 < (v+1)a_1,x_2=a_2(m-1)-1,0\le x_3<a_3m,x_4=0$$ and $$va_1\leq x_1 < (v+1)a_1,x_2=a_2m-1,0\le x_3<a_3m,x_4=0.$$
The basis step $v=0$ has already been done. Now suppose that the statement is true for a given $0\leq v<m-1$. Then we can recover the conjugates with $$(v+1)a_1\leq x_1 < (v+2)a_1,x_2=a_2m-1,0\le x_3<a_3m,x_4=0$$
using the relation $R_4\sum_{u=0}^{m-1}\sigma_1^{a_1u}\sigma_2^{a_2u}\sigma_3^{a_3u}\sim 0$, again due to the fact that for any two different conjugates of $\eta$ used in this relation, the difference of their exponents of $\sigma_2$ is divisible by $a_2$ (and we have already recovered all of them except exactly one) and subsequently the conjugates with $$(v+1)a_1\leq x_1 < (v+2)a_1,x_2=a_2(m-1)-1,0\le x_3<a_3m,x_4=0$$ using the relation $R_2N_2\sim 0$.
Therefore the induction is complete and we have recovered all the conjugates of $\eta$. 

Thus we have proven the following theorem:
\begin{theorem}\label{th1}
Under the assumptions on page \pageref{assum}, if $r_1=r_2=r_3=r_4=1$, then the set $B_{1}\cup B_D$ forms a basis of $D^+$ and the set $B_{1}\cup B_C$ forms a basis of $C^+$.
\end{theorem}
\section{The case $r_1=r_2=a_3=r_4=1$}
In this case, we have
$$\Gal(k/\Qbb)\cong
 \{\restr{\sigma_1^{x_1}\sigma_2^{x_2}\sigma_3^{x_3}\sigma_4^{x_4}}{k};~  0\leq x_1<a_1m, 0\leq x_2<a_2m, 0\leq x_3<n_3,0\leq x_4<a_4\},$$
  $$s_{12}=s_{13}=s_{14}=s_{23}=s_{24}=s_{34}=1,$$
$$R_1N_1\sim 0, R_2N_2\sim 0, N_3 \sim 0, R_4 \sum_{u=0}^{m-1}\sigma_1^{a_1u}\sigma_2^{a_2u}\sigma_3^{u} \sim 0$$
and
\begin{align*}
N&=a_1a_2a_4\frac{m^3}{r_3}-a_1a_2a_4m^2-(a_1a_2+a_1a_4+a_2a_4)\frac{m^2}{r_3}\\
&\hp+(a_1a_2+a_1a_4+a_2a_4)m+(a_1+a_2+a_4)\frac{m}{r_3}-a_1-a_2-a_4\\
&=(n_3-1)(a_1a_2a_4m^2-(a_1a_2+a_1a_4+a_2a_4)m+a_1+a_2+a_4)\\
&=(n_3-1)(a_1a_2m^2-(a_1a_2+a_1+a_2)m+a_1+a_2+1)\\
&\hp+(n_3-1)(a_4-1)(a_1a_2m^2-a_1m-a_2m+1)\\
%&=(n_3-1)((a_1m-1)a_2(m-2)\!+\!(a_1m-1)(a_2-1)+\!a_1\!+\!(a_4-1)(a_1m-1)(a_2m-1))\\
&=(n_3-1)(a_4-1)(a_1m-1)(a_2m-1)+(n_3-1)(a_1m-1)(a_2(m-1)-1)\\
&\hp+(n_3-1)a_1.
\end{align*}
%\newpage
We will define $B_2$ as the set of the following $N$ conjugates $\eta^{\sigma_1^{x_1}\sigma_2^{x_2}\sigma_3^{x_3}\sigma_4^{x_4}}$:
\begin{itemize}
\item $0\le x_1 <a_1m-1$, $ 0\le x_2 < a_2m-1$, $ 0 \le x_3<n_3-1$, $ 1\le x_4<a_4$,
\item $0\le x_1 < a_1m-1$, $ 0\le x_2 < a_2(m-1)-1$, $ 1\leq x_3 < n_3$, $ x_4=0$,
\item $0\le x_1 < a_1$, $ x_2=a_2(m-1)-1$, $, 1\leq x_3 < n_3$, $ x_4=0$.
\end{itemize}

First we will recover the cases $0<x_4<a_4$, $x_1=a_1m-1$ or $x_2=a_2m-1$ or $x_3=n_3-1$ using the relations $R_1N_1\sim 0$, $ R_2N_2\sim 0$, $ N_3\sim 0$. From now on, we only need to deal with the cases where $x_4=0$.
\paragraph*{}
Next, we will recover the cases $$x_1=a_1m-1, 0\le x_2 < a_2(m-1)-1, 1\leq x_3 <n_3, x_4=0$$ using the relation $R_1N_1\sim 0$ and subsequently the cases $$0\leq x_1 < a_1m,  0\le x_2 < a_2(m-1)-1, x_3 = x_4=0$$ and $$0\le x_1 < a_1, x_2 = a_2(m-1)-1, x_3 = x_4=0$$ using the relation $N_3\sim 0$.

\paragraph*{}
At this moment, we are only missing the conjugates $\eta^{\sigma_1^{x_1}\sigma_2^{x_2}\sigma_3^{x_3}}$ with $$0\le x_1< a_1m, a_2(m-1)\le x_2 <a_2m, 0\le x_3<n_3$$ and $$a_1\le x_1< a_1m, x_2=a_2(m-1)-1, 0\le x_3<n_3.$$
Next, we will recover all the cases $$0\leq x_1 < a_1m, a_2(m-1)\leq x_2 < a_2m-1, 0\leq x_3 <n_3,x_4=0$$ %and (note that $2a_1\leq a_1m$, since $m>1$) $$a_1\leq x_1 < 2a_1, x_2=0, 0\leq x_3 <n_3,x_4=0$$
using the relation $R_4\sum_{u=0}^{m-1}\sigma_1^{a_1u}\sigma_2^{a_2u}\sigma_3^u\sim 0$, due to the fact that %for any two different conjugates of $\eta$ used in this relation, 
the exponents of $\sigma_2$ in this relation are pairwise congruent modulo $a_2$ (and we have already recovered all of them except exactly one). %differs by at least $a_2$ in their second coordinate.
 After this, we can recover the cases $$0\leq x_1 < a_1, x_2 = a_2m-1, 0\leq x_3 <n_3$$ using the relation $R_2N_2\sim 0$.
\paragraph*{}
Finally, we will use induction with respect to $v=0,1,\dots, m-1$ to show that we can recover the conjugates $\eta^{\sigma_1^{x_1}\sigma_2^{x_2}\sigma_3^{x_3}}$ with $$va_1\leq x_1 < (v+1)a_1,x_2=a_2(m-1)-1,0\le x_3<n_3,x_4=0$$ and $$va_1\leq x_1 < (v+1)a_1,x_2=a_2m-1,0\le x_3<n_3,x_4=0.$$
The basis step $v=0$ has already been done. Now suppose that the statement is true for a given $0\leq v<m-1$. Then we can recover the conjugates with $$(v+1)a_1\leq x_1 < (v+2)a_1,x_2=a_2m-1,0\le x_3<n_3,x_4=0$$
using the relation $R_4\sum_{u=0}^{m-1}\sigma_1^{a_1u}\sigma_2^{a_2u}\sigma_3^u\sim 0$, again due to the fact that for any two different conjugates of $\eta$ used in this relation, the difference of their exponents of $\sigma_2$ is divisible by $a_2$ (and we have already recovered all of them except exactly one) and subsequently the conjugates with $$(v+1)a_1\leq x_1 < (v+2)a_1,x_2=a_2(m-1)-1,0\le x_3<n_3,x_4=0$$ using the relation $R_2N_2\sim 0$.
Therefore the induction is complete and we have recovered all the conjugates of $\eta$. 

Thus we have proven the following theorem:
\begin{theorem}\label{th2}
Under the assumptions on page \pageref{assum}, if $r_1=r_2=a_3=r_4=1$, then the set $B_{2}\cup B_D$ forms a basis of $D^+$ and the set $B_{2}\cup B_C$ forms a basis of $C^+$.
\end{theorem}
\section{The case $a_1=a_2=r_3=r_4=1$}
In this case, we have
$$\Gal(k/\Qbb)\cong
 \{\restr{\sigma_1^{x_1}\sigma_2^{x_2}\sigma_3^{x_3}\sigma_4^{x_4}}{k};~  0\leq x_1<n_1, 0\leq x_2<n_2, 0\leq x_3<a_3m,0\leq x_4<a_4\},$$
  $$s_{12}=\gcd(r_1,r_2),s_{13}=s_{14}=s_{23}=s_{24}=s_{34}=1$$
 and $$N_1\sim 0, N_2\sim 0, R_3N_3 \sim 0, R_4 \sum_{u=0}^{m-1}\sigma_1^u\sigma_2^u\sigma_3^{a_3u} \sim 0.$$

Moreover, by Lemma \ref{coprime}, we have $s_{12}\frac{m}{r_1r_2}=\gcd(n_1,n_2)$, hence
\begin{align*}
N&=a_3a_4\frac{m^3}{r_1r_2}-a_3\frac{m^2}{r_1r_2}-a_4\frac{m^2}{r_1r_2}-a_3a_4\left(\frac{m^2}{r_1}+\frac{m^2}{r_2}\right)+s_{12}\frac{m}{r_1r_2}\\
&\hp+a_3(n_1+n_2)+a_4(n_1+n_2)+a_3a_4m-a_3-a_4-1\\
%&=a_3\left( \frac{m^3}{r_1r_2}-\frac{m^2}{r_1r_2} -\frac{m^2}{r_1}-\frac{m^2}{r_2}+n_1+n_2+m-1\right)+\frac{m^2}{r_1r_2}+n_1+n_2-1+
%\gcd(n_1,n_2)
&=a_3\left(mn_1n_2-n_1n_2 -mn_1-mn_2+n_1+n_2+m-1\right)-n_1n_2+n_1+n_2-1\\
&\hp+(a_4-1)(a_3\left(mn_1n_2 -mn_1-mn_2+m\right)-n_1n_2+n_1+n_2-1)+\gcd(n_1,n_2)-1\\
&=a_3\left(m-1\right)\left(n_1-1\right)\left(n_2-1\right)-\left(n_1-1\right)\left(n_2-1\right)\\
&\hp+(a_4-1)\left(a_3m\left(n_1-1\right)\left(n_2-1\right)-\left(n_1-1\right)\left(n_2-1\right)\right)+\gcd(n_1,n_2)-1\\
&=(n_1-1)(n_2-1)(a_3(m-1)-1)\\
&\hp+(a_4-1)(n_1-1)(n_2-1)(a_3m-1)+\gcd(n_1,n_2)-1.
\end{align*}

We will define $B_3$ as the set of the following $N$ conjugates $\eta^{\sigma_1^{x_1}\sigma_2^{x_2}\sigma_3^{x_3}\sigma_4^{x_4}}$:
\begin{itemize}
\item $0\leq x_1<n_1-1$, $ 0\leq x_2<n_2-1$, $ 0\leq x_3<a_3m-1$, $ 0<x_4\leq a_4-1$,
\item $0\leq x_1<n_1-1$, $ 0\leq x_2<n_2-1$, $ a_3<x_3<a_3m$, $ x_4=0,$
\item $1\leq x_1<\gcd(n_1,n_2)$, $ x_2=0$, $ x_3=0$, $x_4=0$.
\end{itemize}

\paragraph*{}
First we will recover the cases $0<x_4<a_4$, $x_1=n_1-1$ or $x_2=n_2-1$ or $x_3=a_3m-1$ using the relations $N_1\sim 0$, $ N_2\sim 0$, $ R_3N_3\sim 0$. From now on, we only need to deal with the cases where $x_4=0$.
\paragraph*{}
Next, we will recover the cases $x_4=0$, $a_3< x_3<a_3m$, $x_1=n_1-1$ or $x_2=n_2-1$ using the relations $N_1\sim 0$, $ N_2\sim 0$. Now we can also recover the cases $$0\leq x_1<n_1, 0\leq x_2< n_2, 1\leq x_3<a_3, x_4=0$$
using the relation $R_4\sum_{u=0}^{m-1}\sigma_1^{u}\sigma_2^{u}\sigma_3^{a_3u}\sim 0$, due to the fact that the exponents of $\sigma_3$ in this relation are pairwise congruent modulo $a_3$ (and we have already recovered all of them except exactly one).
\paragraph*{}
At this moment, we are only missing the conjugates $\eta^{\sigma_1^{x_1}\sigma_2^{x_2}\sigma_3^{a_3}}$ for all $$0\leq x_1<n_1, 0\leq x_2<n_2$$ 
and among the conjugates $\eta^{\sigma_1^{x_1}\sigma_2^{x_2}}$
we have only those with 
$0 < x_1 <gcd(n_1,n_2),  x_2 = 0$.
%and $\eta^{\sigma_1^{x_1}\sigma_2^{x_2}}$ for all $$0\leq x_1<n_1, 1\leq x_2<n_2 \text{ and } \gcd(n_1,n_2)\leq x_1<n_1,x_2=0 \text{ and } x_1=x_2=0.$$
We will focus on recovering the remaining conjugates $\eta^{\sigma_1^{x_1}\sigma_2^{x_2}}$, 
%those of the second kind, 
because once we have those, we can recover those with $x_3=a_3$, $x_4=0$ just by using the relation $R_3N_3\sim 0$.

\paragraph*{}
Let $Q'$ be the quotient $\Z[G]$-module $$D^+/\big\langle \{\eta_I \big\vert \emptyset \subsetneq I \subsetneq P \}\big\rangle_{\Z[G]},$$
so that $Q'$ is generated by the class of $\eta$ as a $\Z[G]$-module. Furthermore, let $Q$ be the quotient $\Z$-module of $Q'$ by the classes of conjugates we have already recovered, i.e., %if we denote the class of $\eta$ in $Q'$ again by $\eta$, we have
\begin{align*}
Q:=Q'/\big\langle \{\eta^{\sigma_1^{x_1}\sigma_2^{x_2}\sigma_3^{x_3}\sigma_4^{x_4}}; \quad & 0\leq x_1< n_1, 0\leq x_2<n_2, 0\leq x_3<a_3m,0< x_4<a_4,\\
\text { or }& 0\leq x_1< n_1, 0\leq x_2<n_2, 1\leq x_3<a_3m, x_3\neq a_3, x_4=0,\\
\text { or }& 1\leq x_1<\gcd(n_1,n_2), x_2=x_3=x_4=0 \}\big\rangle_{\Z}
\end{align*}
(where we denote $\eta^{\rho}\in D^+$ and its class in $Q'$ in the same way for any $\rho\in G$).
We will write $Q$ additively, denoting the class of $\eta$ in $Q$ by $\mu$, hence denoting the class of $\eta^{\rho}$ in $Q$ by $\rho\cdot \mu$ for any $\rho\in\Gal(k/\Q)$ or $\rho\in G$.
Showing that we have indeed chosen a basis now amounts to showing that $Q$ is trivial. Since $$0=\sigma_1^{x_1}\sigma_2^{x_2}R_3N_3\cdot \mu=\sigma_1^{x_1}\sigma_2^{x_2}\cdot \mu+\sigma_1^{x_1}\sigma_2^{x_2}\sigma_3^{a_3}\cdot \mu$$
for any $x_1,x_2\in\Z$,
this is equivalent with showing that $\sigma_1^{x_1}\sigma_2^{x_2}\cdot \mu=0$ for any $0\leq x_1<n_1$, $0\leq x_2<n_2$.

\paragraph*{}
\begin{lemma}\label{diag}
In $Q$, we have $$\sigma_1^{x_1}\sigma_2^{x_2}(1-\sigma_1\sigma_2)\cdot \mu=0$$
for any $x_1,x_2\in\Z$.
\end{lemma}
\begin{proof}
Using the fact that the order of $\sigma_3$ is $a_3m$, we have
\begin{align*}
0&\sim \sigma_1^{x_1}\sigma_2^{x_2}\left(R_3R_4\sum_{u=0}^{m-1}\sigma_1^{u}\sigma_2^{u}\sigma_3^{a_3u}-\sigma_1\sigma_2R_4R_3N_3\right)\\
&=\sigma_1^{x_1}\sigma_2^{x_2}R_3R_4\sum_{u=0}^{m-1}\left(\sigma_1^{u}\sigma_2^{u}-\sigma_1\sigma_2\right)\sigma_3^{a_3u}\\
&=\sigma_1^{x_1}\sigma_2^{x_2}(1-\sigma_1\sigma_2)R_3R_4+\sigma_1^{x_1}\sigma_2^{x_2}R_3R_4\sum_{u=2}^{m-1}\left(\sigma_1^{u}\sigma_2^{u}-\sigma_1\sigma_2\right)\sigma_3^{a_3u}\\
&=\sigma_1^{x_1}\sigma_2^{x_2}(1-\sigma_1\sigma_2)+
%\underbrace{
\sigma_1^{x_1}\sigma_2^{x_2}(1-\sigma_1\sigma_2)R_3\sum_{u=1}^{a_4-1}\sigma_4^u
%}_{
%\text{contains only terms with } x_4>0
%=0, \text{ as explained below}
%}
\\
&\hp+
%\underbrace{
\sigma_1^{x_1}\sigma_2^{x_2}(1-\sigma_1\sigma_2)\sum_{u=1}^{a_3-1}\sigma_3^u+\sigma_1^{x_1}\sigma_2^{x_2}R_3R_4\sum_{u=2}^{m-1}\left(\sigma_1^{u}\sigma_2^{u}-\sigma_1\sigma_2\right)\sigma_3^{a_3u}
%}_{
%\text{contains only terms with } 1\leq x_3<a_3m \text{ and } x_3\neq a_3
%=0, \text{ as explained below}
%}.
\end{align*}
Since all the summands in the expression 
\begin{align*}
&\sigma_1^{x_1}\sigma_2^{x_2}(1-\sigma_1\sigma_2)R_3\sum_{u=1}^{a_4-1}\sigma_4^u
+\sigma_1^{x_1}\sigma_2^{x_2}(1-\sigma_1\sigma_2)\sum_{u=1}^{a_3-1}\sigma_3^u\\
+&\sigma_1^{x_1}\sigma_2^{x_2}R_3R_4\sum_{u=2}^{m-1}\left(\sigma_1^{u}\sigma_2^{u}-\sigma_1\sigma_2\right)\sigma_3^{a_3u}
\end{align*}
have either $x_4>0$ or $ 1\leq x_3<a_3m,  x_3\neq a_3$ (where $x_3$ and $x_4$ denote the respective exponents of $\sigma_3$ and $\sigma_4$ in each term), the result of their action on $\mu$ becomes trivial in $Q$, which yields the result.
\end{proof}

\begin{lemma}\label{M0}
For any $0\leq x_1<n_1, 0\leq x_2<n_2$, we have $$\sigma_1^{x_1}\sigma_2^{x_2}\cdot \mu=\begin{cases}
\mu \quad \text{ if } x_1\equiv x_2 \pmod{\gcd(n_1,n_2)},\\
0 \quad \text{ otherwise}.
\end{cases}$$
\end{lemma}
\begin{proof}
First we will prove that for any $1\leq u<\gcd(n_1,n_2)$ and $0\leq v<\lcm(n_1,n_2)$, we have 
\begin{equation}\label{UVV}
\sigma_1^{u+v}\sigma_2^{v}\cdot \mu=0
\end{equation}
and
\begin{equation}\label{VV}
\sigma_1^{v}\sigma_2^{v}\cdot \mu=\mu.
\end{equation}
We will do so simultaneously by induction on $v$. For $v=0$, this follows directly from the definitions of $Q$ and $\mu$. Now suppose that the statements hold for $0\leq v<\lcm(n_1,n_2)-1$. Then Lemma \ref{diag} implies that $$\sigma_1^{u+(v+1)}\sigma_2^{v+1}\cdot \mu=\sigma_1^{u+v}\sigma_2^{v}\cdot \mu=0$$
and $$\sigma_1^{v+1}\sigma_2^{v+1}\cdot \mu=\sigma_1^{v}\sigma_2^{v}\cdot \mu=\mu$$
by the induction hypothesis, so both statements also hold for $v+1$ and we are done with the induction.

\paragraph*{}
Now consider the map $$\{0,1,\dots,\gcd(n_1,n_2)\}\times \{0,1,\dots,\lcm(n_1,n_2)\} \to \{0,1,\dots,n_1\}\times \{0,1,\dots,n_2\}$$
given by $(u,v)\mapsto (u+v\pmod{n_1},v \pmod{n_2})$. Suppose that both $(u,v)$ and $(u',v')$ map to the same element. Then, for suitable $q, q' \in \Z$, $$(u-u')+(v-v')=qn_1$$ and $$v-v'=q'n_2,$$ hence $$u-u'=qn_1-q'n_2\equiv 0 \pmod {\gcd(n_1,n_2)}.$$ Since $0\le u,u'\le \gcd(n_1,n_2)$, this implies $u=u'$. Consequently both $n_1$ and $n_2$ divide $v-v'$, hence so does $\lcm(n_1,n_2)$ and $v=v'$ (using that $0\le v,v'\le \lcm(n_1,n_2)$). Thus we have shown that the above map is injective, and since both sets have cardinality $n_1n_2$, it must be a bijection. Therefore for any $0\leq x_1<n_1$, $0\leq x_2<n_2$, we can write $$\sigma_1^{x_1}\sigma_2^{x_2}\cdot \mu=\sigma_1^{u+v}\sigma_2^{v}\cdot \mu$$
for unique $0\leq u<\gcd(n_1,n_2)$ and $0\leq v<\lcm(n_1,n_2)$, and the equalities \eqref{UVV} and \eqref{VV} imply that $\sigma_1^{x_1}\sigma_2^{x_2}\cdot \mu=0$ unless $u=0$, in which case $\sigma_1^{x_1}\sigma_2^{x_2}\cdot \mu=\mu$. But the congruences $$x_1\equiv u+v \pmod{n_1}$$ and $$x_2\equiv v \pmod{n_2}$$ imply that $$x_1-x_2\equiv u  \pmod{(\gcd(n_1,n_2)},$$
so the condition $u=0$ is equivalent to $$x_1\equiv x_2 \pmod{(\gcd(n_1,n_2)},$$ as needed.
% (note that this is well defined, since the order of $\sigma_1$ is $n_1$ and the order of $\sigma_2$ is $n_2$).

%Now note that any $0\leq x_1<n_1$, $0\leq x_2<n_2$, we can uniquely write $$x_1-x_2=v\cdot \gcd(n_1,n_2)+u,$$ where $0\leq u<\gcd(n_1,n_2)$ and $0\leq v<n_1\leq \lcm(n_1,n_2)$. Hence $$\sigma_1^{x_1}\sigma_2^{x_2}\cdot\mu=\sigma_1^{v\cdot \gcd(n_1,n_2)+u+x_2}\sigma_2^{x_2}\cdot\mu$$
\end{proof}

\begin{prop}
We have $\mu=0.$
\end{prop}
\begin{proof}
Using the relation $N_1\sim 0$ and Lemma \ref{M0} together with the bijection $$\{0,1,\dots,\gcd(n_1,n_2)-1\}\times \{0,1,\dots,\frac{n_1}{\gcd(n_1,n_2)}-1\}\to \{0,1,\dots, n_1-1\}$$
given by $(u,v)\mapsto v\cdot \gcd(n_1,n_2)+u$, we get
\begin{align*}
0&=N_1\cdot \mu=\sum_{w=0}^{n_1-1}\sigma_1^w\cdot \mu=\sum_{u=0}^{\gcd(n_1,n_2)-1}\sum_{v=0}^{\frac{n_1}{\gcd(n_1,n_2)}-1}\sigma_1^{ v\cdot \gcd(n_1,n_2)+u}\cdot \mu
%&=\sum_{u=0}^{\gcd(n_1,n_2)-1}\sum_{v=0}^{\frac{n_1}{\gcd(n_1,n_2)}-1}\sigma_1^{u}\cdot \mu\\
%&=\sum_{v=0}^{\frac{n_1}{\gcd(n_1,n_2)}-1} \mu+\sum_{u=1}^{\gcd(n_1,n_2)-1}\sum_{v=0}^{\frac{n_1}{\gcd(n_1,n_2)}-1}\underbrace{\sigma_1^{u}\cdot \mu}_{=0 \text { by Lemma \ref{M0} }}
=\frac{n_1}{\gcd(n_1,n_2)}\cdot \mu,
\end{align*}
since by Lemma \ref{M0}, $\sigma_1^{ v\cdot \gcd(n_1,n_2)+u}\cdot \mu$ is zero for $u\neq 0$ and equal to $\mu$ otherwise.

Analogously, we get 
\begin{align*}
0&=N_2\cdot \mu=\sum_{w=0}^{n_2-1}\sigma_2^w\cdot \mu=\sum_{u=0}^{\gcd(n_1,n_2)-1}\sum_{v=0}^{\frac{n_2}{\gcd(n_1,n_2)}-1}\sigma_2^{ v\cdot \gcd(n_1,n_2)+u}\cdot \mu
%&=\sum_{u=0}^{\gcd(n_1,n_2)-1}\sum_{v=0}^{\frac{n_2}{\gcd(n_1,n_2)}-1}\sigma_2^{u}\cdot \mu\\
%&=\sum_{v=0}^{\frac{n_2}{\gcd(n_1,n_2)}-1} \mu+\sum_{u=1}^{\gcd(n_1,n_2)-1}\sum_{v=0}^{\frac{n_2}{\gcd(n_1,n_2)}-1}\underbrace{\sigma_2^{u}\cdot \mu}_{=0 \text { by Lemma \ref{M0} }}\\
=\frac{n_2}{\gcd(n_1,n_2)}\cdot \mu,
\end{align*}
since by Lemma \ref{M0}, $\sigma_2^{ v\cdot \gcd(n_1,n_2)+u}\cdot \mu$ is zero for $u\neq 0$ and equal to $\mu$ otherwise.

Due to the fact that $\frac{n_1}{\gcd(n_1,n_2)}$ and $\frac{n_2}{\gcd(n_1,n_2)}$ are coprime, this implies $\mu=0$ by Bézout's identity.
\end{proof}
It now follows that $Q$ is trivial, so we have proven the following theorem:
\begin{theorem}\label{th3}
Under the assumptions on page \pageref{assum}, if $a_1=a_2=r_3=r_4=1$, then the set $B_{3}\cup B_D$ forms a basis of $D^+$ and the set $B_{3}\cup B_C$ forms a basis of $C^+$.
\end{theorem}

\section{The case $a_1\!=\!a_2\!=\!a_3\!=\!r_4\!=\!1$, $\gcd(n_1,n_2,n_3)\!=\!\gcd(n_1,n_2)$}
In this case, we have
$$\Gal(k/\Qbb)\cong
 \{\restr{\sigma_1^{x_1}\sigma_2^{x_2}\sigma_3^{x_3}\sigma_4^{x_4}}{k};~  0\leq x_1<n_1, 0\leq x_2<n_2, 0\leq x_3<n_3,0\leq x_4<a_4\},$$
  $$s_{12}=\gcd(r_1,r_2), s_{13}=\gcd(r_1,r_3), s_{23}=\gcd(r_2,r_3), s_{14}=s_{24}=s_{34}=1$$
 and
$$N_1\sim 0, N_2\sim 0, N_3 \sim 0, R_4 \sum_{u=0}^{m-1}\sigma_1^{u}\sigma_2^{u}\sigma_3^{u} \sim 0.$$

\begin{lemma}\label{LCM3}
For any $a,b,c\in\Nbb$, we have $$\lcm(a,b,c)=\frac{abc\cdot \gcd(a,b,c)}{\gcd(a,b)\cdot \gcd(a,c)\cdot\gcd(b,c)}.$$
\end{lemma}
\begin{proof}
Let $d:=\gcd(a,b,c)$. Then there exist $a',b',c'\in\Z$ such that $$a=da', b=db', c=dc',\gcd(a',b',c')=1.$$ Letting $$e:=\gcd(a',b'), f:=\gcd(a',c'), g:=\gcd(b',c'),$$ we get that there must exist $a'',b'',c''\in\Z$ such that $$a=defa'', b=degb'', c=dfgc''$$ and $$\gcd(a'',b'')=\gcd(a'',c'')=\gcd(b'',c'')=1.$$ Also the condition $\gcd(a',b',c')=1$ can be reformulated as  $$\gcd(e,f)=\gcd(e,g)=\gcd(f,g)=1.$$
Thus we get $$\lcm(a,b,c)=defga''b''c''=\frac{abcdefg}{d^3e^2f^2g^2}=\frac{abc\cdot d}{de\cdot df\cdot dg}=\frac{abc\cdot \gcd(a,b,c)}{\gcd(a,b)\cdot \gcd(a,c)\cdot\gcd(b,c)},$$
as needed.
\end{proof}

\begin{lemma}\label{TFAE4}
%Under the assumptions $s_{12}=s_{13}=s_{23}=1,$ 
The following are equivalent:
\begin{enumerate}[label={\upshape(\roman*)}]
\item $\gcd(n_1,n_2,n_3)=\gcd(n_1,n_2)$,
\item $\frac{n_1n_2n_3}{m}=\gcd(n_1,n_3)\cdot \gcd(n_2,n_3)$,
\item $\gcd(n_1,n_2)=\gcd(\gcd(n_1,n_3),\gcd(n_2,n_3))$,
\item $\gcd(n_1,n_2) \mid n_3$.
\end{enumerate}
\end{lemma}
\begin{proof}
\leavevmode
%mbox{}\\
\begin{DESCRIPTION}%[labelindent=0cm]
\item[\enquote{(i) $\Leftrightarrow$ (ii)}:] Using Lemma \ref{LCM3} together with Lemma \ref{coprime}, we get 
$$m=\lcm(n_1,n_2,n_3)=\frac{n_1n_2n_3\cdot \gcd(n_1n_2n_3)}{\gcd(n_1,n_2)\cdot \gcd(n_1,n_3)\cdot\gcd(n_2,n_3)},$$
hence $$\frac{n_1n_2n_3}{m}=\frac{\gcd(n_1,n_2)\cdot \gcd(n_1,n_3)\cdot\gcd(n_2,n_3)}{\gcd(n_1n_2n_3)}$$
and this equals $\gcd(n_1,n_3)\cdot\gcd(n_2,n_3)$ iff $\gcd(n_1,n_2,n_3)=\gcd(n_1,n_2)$.
\item[\enquote{(i) $\Leftrightarrow$ (iii)}:] It suffices to show that $\gcd(n_1,n_2,n_3)=\gcd(\gcd(n_1,n_3),\gcd(n_2,n_3))$. This is true in general, because any integer is a common divisor of $n_1,n_2,n_3$ iff it is a common divisor of $n_1,n_3$ and a common divisor of $n_2,n_3$ iff it is a common divisor of $\gcd(n_1,n_3)$ and $\gcd(n_2,n_3)$. %iff it is a divisor of $\gcd(\gcd(n_1,n_3),\gcd(n_2,n_3))$.
\item[\enquote{(i) $\Rightarrow$ (iv)}:] This is immediate, because $\gcd(n_1,n_2,n_3)$ is a divisor of $n_3$.
\item[\enquote{(i) $\Leftarrow$ (iv)}:] The relation $\gcd(n_1,n_2,n_3) \mid \gcd(n_1,n_2)$ is true automatically. Conversely, if $\gcd(n_1,n_2) \mid n_3$, then $\gcd(n_1,n_2)$ is a common divisor of $n_1,n_2,n_3$, hence also $\gcd(n_1,n_2) \mid \gcd(n_1,n_2,n_3)$.
\end{DESCRIPTION}
\end{proof}

Therefore, using Lemma \ref{TFAE4} together with Lemma \ref{coprime}, we get
\begin{align*}
N&=a_4n_1n_2n_3-\frac{n_1n_2n_3}{m}-a_4(n_1n_2+n_1n_3+n_2n_3)-a_4-2+a_4(n_1+n_2+n_3)\\
&\hp+\gcd(n_1,n_2)+\gcd(n_1,n_3)+\gcd(n_2,n_3)\\
&=(a_4-1)(n_1-1)(n_2-1)(n_3-1)+(n_1-1)(n_2-1)(n_3-2)+(n_1-1)(n_2-1)-2\\
&\hp-\gcd(n_1,n_3)\cdot\gcd(n_2,n_3)+\gcd(n_1,n_2)+\gcd(n_1,n_3)+\gcd(n_2,n_3)\\
&=(a_4-1)(n_1-1)(n_2-1)(n_3-1)+(n_1-1)(n_2-1)(n_3-2)\\
&\hp+(n_1-1)(n_2-\gcd(n_2,n_3))+(n_1-\gcd(n_1,n_3))\cdot(\gcd(n_2,n_3)-1)\!+\!\gcd(n_1,n_2)\!-\!1.
\end{align*}
(Note that $n_3=a_3n_3>1$ by Remark \ref{remAN}.)

We will define $B_4$ as the set of the following $N$ conjugates $\eta^{\sigma_1^{x_1}\sigma_2^{x_2}\sigma_3^{x_3}\sigma_4^{x_4}}$:
\begin{itemize}
\item $0\leq x_1<n_1-1$, $ 0\leq x_2<n_2-1$, $ 0\leq x_3<n_3-1$, $ 0<x_4\leq a_4-1$,
\item $0\leq x_1<n_1-1$, $ 0\leq x_2<n_2-1$, $ 1< x_3 \leq n_3-1$, $ x_4=0$,
\item $1\leq x_1< n_1$, $ \gcd(n_2,n_3)\leq x_2<n_2$, $ x_3=0$, $ x_4=0$,
\item $\gcd(n_1,n_3)\leq x_1< n_1$, $ 1\leq x_2<\gcd(n_2,n_3)$, $x_3=0$, $ x_4=0$,
\item $1\leq x_1 <\gcd(n_1,n_2)$, $ x_2=0$, $ x_3=0$, $ x_4=0$.
\end{itemize}

\paragraph*{}
First we will recover the cases $0<x_4<a_4$, $x_1=n_1-1$ or $x_2=n_2-1$ or $x_3=n_3-1$ using the relations $N_1\sim 0$, $N_2\sim 0$, $N_3\sim 0$. From now on, we only need to deal with the cases where $x_4=0$. Next, we will recover the cases $1< x_3 \leq n_3-1$, $x_1=n_1-1$ or $x_2=n_2-1$ (and always $x_4=0$) using the relations $N_1\sim 0$, $N_2\sim 0$ and the cases $x_2=x_3=x_4=0$, $\gcd(n_1,n_3)\leq x_1< n_1$ and $x_3=x_4=0$, $x_1=0, \gcd(n_2,n_3)\leq x_2<n_2$ using the relations $N_2\sim 0$, $N_1\sim 0$.
\paragraph*{}
At this moment, we are only missing all the cases with $x_3=1,x_4=0$ and some of those with $x_3=x_4=0$. From now on, we will only focus on recovering those with $x_3=x_4=0$, because once we have those, we can recover those with $x_3=1$, $x_4=0$ just by using the relation $N_3\sim 0$.

\paragraph*{}
Let $Q'$ be the quotient $\Z[G]$-module $$D^+/\big\langle \{\eta_I \big\vert \emptyset \subsetneq I \subsetneq P \}\big\rangle_{\Z[G]}$$
and let $Q$ be the quotient $\Z$-module of $Q'$ by the classes of conjugates we have already recovered, i.e.,
\begin{align*}
Q:=Q'/\big\langle \{\eta^{\sigma_1^{x_1}\sigma_2^{x_2}\sigma_3^{x_3}\sigma_4^{x_4}}; \quad & 0\leq x_1< n_1, 0\leq x_2<n_2, 0\leq x_3<n_3,0< x_4<a_4,\\
\text { or }& 0\leq x_1< n_1, 0\leq x_2<n_2, 1< x_3<n_3,x_4=0,\\
\text { or }& 0\leq x_1< n_1, \gcd(n_2,n_3)\leq x_2<n_2, x_3=x_4=0,\\
\text { or }& \gcd(n_1,n_3)\leq x_1< n_1, 0\leq x_2<\gcd(n_2,n_3), x_3=x_4=0\\ 
\text { or }&1\leq x_1 <\gcd(n_1,n_2), x_2=x_3=x_4=0 \}\big\rangle_{\Z}
\end{align*}
(where we denote $\eta^{\rho}\in D^+$ and its class in $Q'$ in the same way for any $\rho\in G$).
We will write $Q$ additively, denoting the class of $\eta$ in $Q$ by $\mu$, hence denoting the class of $\eta^{\rho}$ in $Q$ by $\rho\cdot \mu$ for any $\rho\in\Gal(k/\Q)$ or $\rho\in G$.
Showing that we have indeed chosen a basis now amounts to showing that $Q$ is trivial. Since $$0=\sigma_1^{x_1}\sigma_2^{x_2}N_3\cdot \mu=\sigma_1^{x_1}\sigma_2^{x_2}\cdot \mu+\sigma_1^{x_1}\sigma_2^{x_2}\sigma_3\cdot \mu$$
for any $x_1,x_2\in\Z$,
this is equivalent with showing that $\sigma_1^{x_1}\sigma_2^{x_2}\cdot \mu=0$ for each $0\leq x_1<n_1$, $0\leq x_2<n_2$ (and because of the definition of $Q$, it suffices to show this only for each $0\leq x_1<\gcd(n_1,n_3)$, $0\leq x_2<\gcd(n_2,n_3)$).

\paragraph*{}
The conjugates with $x_3=0$ and $x_4=0$ (i.e., those of the form $\eta^{\sigma_1^{x_1}\sigma_2^{x_2}}$) can be visualized as a discrete rectangle with $n_1$ rows and $n_2$ columns. Since for each $x_4$, there are $n_3$ layers of such rectangles in total, the sum $\eta^{R_4\sum_{u=0}^{m-1}\sigma_1^{u}\sigma_2^{u}\sigma_3^{u}}$ must contain $\frac{m}{n_3}=r_3$ conjugates in each of these rectangles (and in this case, it can be seen geometrically that these form a regular grid). We will now describe the sum of these. 

Let $$T:=\sum_{u=0}^{r_3-1}\sigma_1^{un_3}\sigma_2^{un_3}.$$

\begin{lemma}\label{Tdiag4}
In $Q$, we have $$\sigma_1^{x_1}\sigma_2^{x_2}(1-\sigma_1\sigma_2)T\cdot \mu=0$$
for any $x_1,x_2\in\Z$.
\end{lemma}
\begin{proof}
Using the fact that every $0\leq w<m$ can be uniquely written as $un_3+v$, where $0\leq u<r_3$, $0\leq v<n_3$, together with the fact that the order of $\sigma_3$ is $n_3$, we get 
\begin{align*}
R_4T\sum_{v=0}^{n_3-1}\sigma_1^{v}\sigma_2^{v}\sigma_3^{v}=R_4\sum_{u=0}^{r_3-1}\sigma_1^{un_3}\sigma_2^{un_3}\sigma_3^{un_3}\cdot \sum_{v=0}^{n_3-1}\sigma_1^{v}\sigma_2^{v}\sigma_3^{v}=R_4\sum_{w=0}^{m-1}\sigma_1^{w}\sigma_2^{w}\sigma_3^{w}\sim 0.
\end{align*}
Together with $N_3\sim 0$, this means that 
\begin{align*}
0&\sim \sigma_1^{x_1}\sigma_2^{x_2}\left(R_4T \sum_{v=0}^{n_3-1}\sigma_1^{v}\sigma_2^{v}\sigma_3^{v}-\sigma_1\sigma_2N_3R_4T\right)=\sigma_1^{x_1}\sigma_2^{x_2}R_4T\sum_{v=0}^{n_3-1}\left(\sigma_1^{v}\sigma_2^{v}-\sigma_1\sigma_2\right)\sigma_3^{v}\\
&=\sigma_1^{x_1}\sigma_2^{x_2}(1-\sigma_1\sigma_2)R_4T+\sigma_1^{x_1}\sigma_2^{x_2}R_4T\sum_{v=2}^{n_3-1}\left(\sigma_1^{v}\sigma_2^{v}-\sigma_1\sigma_2\right)\sigma_3^{v}\\
&=\sigma_1^{x_1}\sigma_2^{x_2}(1-\sigma_1\sigma_2)T\!+\!
%\underbrace{
\sigma_1^{x_1}\sigma_2^{x_2}(1\!-\!\sigma_1\sigma_2)T\sum_{u=1}^{a_4-1}\sigma_4^u\!+\!\sigma_1^{x_1}\sigma_2^{x_2}R_4T\sum_{v=2}^{n_3-1}\left(\sigma_1^{v}\sigma_2^{v}\!-\!\sigma_1\sigma_2\right)\sigma_3^{v}.
%}_{
%\text{contains only terms with } x_4>0 \text{ or } x_3>1
%=0, \text{ as explained below}
%}
\end{align*}

Since all the summands in the expression $$\sigma_1^{x_1}\sigma_2^{x_2}(1-\sigma_1\sigma_2)T\sum_{u=1}^{a_4-1}\sigma_4^u+\sigma_1^{x_1}\sigma_2^{x_2}R_4T\sum_{v=2}^{n_3-1}\left(\sigma_1^{v}\sigma_2^{v}-\sigma_1\sigma_2\right)\sigma_3^{v}$$
have either $x_4>0$ or $x_3>1$ (where $x_3$ and $x_4$ denote the respective exponents of $\sigma_3$ and $\sigma_4$ in each term), the result of their action on $\mu$ becomes trivial in $Q$, which yields the result.
\end{proof}


\begin{lemma}\label{M04}
For any $0\leq x_1<\gcd(n_1,n_3)$, $ 0\leq x_2<\gcd(n_2,n_3)$, we have $$\sigma_1^{x_1}\sigma_2^{x_2}\cdot \mu=\begin{cases}
\mu \quad \text{ if } x_1\equiv x_2 \pmod{\gcd(n_1,n_2)},\\
0 \quad \text{ otherwise}.
\end{cases}$$
\end{lemma}
\begin{proof}
Let $0\leq x_1<\gcd(n_1,n_3), 0\leq x_2<\gcd(n_2,n_3)$ be arbitrary. For the same reasons as in the proof of Lemma \ref{M0}, we can write $$\sigma_1^{x_1}\sigma_2^{x_2}\cdot \mu=\sigma_1^{u+v}\sigma_2^{v} \cdot \mu,$$
where $0\leq u<\gcd(n_1,n_2)=\gcd(\gcd(n_1,n_3),\gcd(n_2,n_3))$ (by Lemma \ref{TFAE4}) and $0\leq v<\lcm(n_1,n_2)$. Moreover, by Lemma \ref{Tdiag4}, we have
\begin{align*}
\sigma_1^{u+v}\sigma_2^{v}T\cdot \mu&=\sigma_1^{u+v-1}\sigma_2^{v-1}T\cdot \mu=\dots=\sigma_1^{u+1}\sigma_2T\cdot \mu=\sigma_1^{u}T\cdot \mu\\
&=\sum_{w=0}^{r_3-1}\sigma_1^{wn_3+u}\sigma_2^{wn_3}\cdot \mu=\sigma_1^u\cdot \mu,
\end{align*}
where we used the definition of $Q$ and the fact that $n_3$ is divisible by $$\gcd(n_1,n_2)=\gcd(\gcd(n_1,n_3),\gcd(n_2,n_3))$$ by Lemma \ref{TFAE4}
The assertion now follows from the definition of $Q$, since $$u\equiv x_1-x_2 \pmod{\gcd(n_1,n_2)}.$$
\end{proof}
%\clearpage
%\newpage
\pagebreak
\begin{prop}
We have $\mu=0.$
\end{prop}
%\nopagebreak[4]
\begin{proof}
Recall that we have $$\gcd(n_1,n_2)=\gcd(\gcd(n_1,n_3),\gcd(n_2,n_3))$$ by Lemma \ref{TFAE4}. Using the relation $N_1\sim 0$ and Lemma \ref{M0} together with the bijection $$\{0,1,\dots,\gcd(n_1,n_2)-1\}\times \{0,1,\dots,\frac{\gcd(n_1,n_3)}{\gcd(n_1,n_2)}-1\}\to \{0,1,\dots, \gcd(n_1,n_3)-1\}$$
given by $(u,v)\mapsto v\cdot \gcd(n_1,n_2)+u$, we get (using the definition of $Q$)
\begin{align*}
0&=N_1\cdot \mu=\sum_{w=0}^{\gcd(n_1,n_3)-1}\sigma_1^w\cdot \mu=\sum_{u=0}^{\gcd(n_1,n_2)-1}\sum_{v=0}^{\frac{\gcd(n_1,n_3)}{\gcd(n_1,n_2)}-1}\sigma_1^{ v\cdot \gcd(n_1,n_2)+u}\cdot \mu
%&=\sum_{u=0}^{\gcd(n_1,n_2)-1}\sum_{v=0}^{\frac{n_1}{\gcd(n_1,n_2)}-1}\sigma_1^{u}\cdot \mu\\
%&=\sum_{v=0}^{\frac{n_1}{\gcd(n_1,n_2)}-1} \mu+\sum_{u=1}^{\gcd(n_1,n_2)-1}\sum_{v=0}^{\frac{n_1}{\gcd(n_1,n_2)}-1}\underbrace{\sigma_1^{u}\cdot \mu}_{=0 \text { by Lemma \ref{M0} }}
=\frac{\gcd(n_1,n_3)}{\gcd(n_1,n_2)}\cdot \mu,
\end{align*}
since by Lemma \ref{M0}, $\sigma_1^{ v\cdot \gcd(n_1,n_2)+u}\cdot \mu$ is zero for $u\neq 0$ and is equal to $\mu$ otherwise.

Analogously, we get 
\begin{align*}
0&=N_2\cdot \mu=\sum_{w=0}^{\gcd(n_2,n_3)-1}\sigma_2^w\cdot \mu=\sum_{u=0}^{\gcd(n_1,n_2)-1}\sum_{v=0}^{\frac{\gcd(n_2,n_3)}{\gcd(n_1,n_2)}-1}\sigma_2^{ v\cdot \gcd(n_1,n_2)+u}\cdot \mu
%&=\sum_{u=0}^{\gcd(n_1,n_2)-1}\sum_{v=0}^{\frac{n_2}{\gcd(n_1,n_2)}-1}\sigma_2^{u}\cdot \mu\\
%&=\sum_{v=0}^{\frac{n_2}{\gcd(n_1,n_2)}-1} \mu+\sum_{u=1}^{\gcd(n_1,n_2)-1}\sum_{v=0}^{\frac{n_2}{\gcd(n_1,n_2)}-1}\underbrace{\sigma_2^{u}\cdot \mu}_{=0 \text { by Lemma \ref{M0} }}\\
=\frac{\gcd(n_2,n_3)}{\gcd(n_1,n_2)}\cdot \mu,
\end{align*}
since by Lemma \ref{M0}, $\sigma_2^{ v\cdot \gcd(n_1,n_2)+u}\cdot \mu$ is zero for $u\neq 0$ and is equal to $\mu$ otherwise.

Due to the fact that $\frac{\gcd(n_1,n_3)}{\gcd(n_1,n_2)}$ and $\frac{\gcd(n_2,n_3)}{\gcd(n_1,n_2)}$ are coprime, this implies $\mu=0$ by Bézout's identity.
\end{proof}

It now follows that $Q$ is trivial, so we have proven the following theorem:
\begin{theorem}\label{th4}
Under the assumptions on page \pageref{assum}, if $$a_1=a_2=a_3=r_4=1,\gcd(n_1,n_2,n_3)=\gcd(n_1,n_2),$$ then the set $B_{4}\cup B_D$ forms a basis of $D^+$ and the set $B_{4}\cup B_C$ forms a basis of $C^+$.
\end{theorem}
\section{The case $a_1=a_2=a_3=r_4=1$, $r_1\neq 1$, $r_2\neq 1$, $r_3 \neq 1$, $s_{12}=s_{13}=s_{23}=1$, $\gcd(n_1,n_2,n_3)=1$}
In this case, we have
\begin{equation*}
\Gal(k/\Qbb)\cong
 \{\restr{\sigma_1^{x_1}\sigma_2^{x_2}\sigma_3^{x_3}\sigma_4^{x_4}}{k};  0\leq x_1<n_1, 0\leq x_2<n_2,  0\leq x_3<n_3,0\leq x_4<a_4\},
\end{equation*}
 $$s_{12}=s_{13}=s_{14}=s_{23}=s_{24}=s_{34}=1$$
and $$N_1\sim 0, N_2\sim 0, N_3\sim 0, R_4\sum_{u=0}^{m-1}\sigma_1^u\sigma_2^u\sigma_3^u\sim0.$$
%Note that the condition $\gcd(n_1,n_2,n_3)=1$ is equivalent to $\lcm(r_1,r_2,r_3)=m$, and since we assume that the $r_i$ are pairwise coprime, this is also equivalent to $m=r_1r_2r_3$. Also 
Note that the condition $r_1\neq 1, r_2\neq 1, r_3 \neq 1$ is actually not restrictive, since we have already discussed the cases where it is not true earlier in this chapter. %Also $r_1,r_2,r_3$ must be pairwise distinct, otherwise their coprimality would imply that two of them equal $1$. %This means that we can without loss of generality assume that $r_1>r_2>r_3$ (MAYBE NOT NEEDED?). 

\begin{lemma}\label{TFAE5}
Under the assumptions $s_{12}=s_{13}=s_{23}=1,$ the following are equivalent:
\begin{enumerate}[label={\upshape(\roman*)}]
\item $\gcd(n_1,n_2,n_3)=1$,
\item $\lcm(r_1,r_2,r_3)=m$,
\item $r_1r_2r_3=m$,
\item $n_1=r_2r_3, n_2=r_1r_3, n_3=r_1r_2$,
\item $\frac{n_1n_2n_3}{m}=m$,
\item $\gcd(n_1,n_2)=r_3, \gcd(n_1,n_3)=r_2, \gcd(n_2,n_3)=r_1$.
\end{enumerate}
\end{lemma}
\begin{proof}
\leavevmode
%mbox{}\\
(Recall that by Lemma \ref{coprime}, we have $s_{ij}=\gcd(r_i,r_j)$.)
\begin{DESCRIPTION}%[labelindent=0cm]
\item[\enquote{(i) $\Leftrightarrow$ (ii)}:] For any $t\in\Z$, we have
\begin{align*}
t \mid \gcd(n_1,n_2,n_3) &\Leftrightarrow t \mid n_1, t\mid n_2, t\mid n_3 \Leftrightarrow r_1\mid \frac{m}{t}, r_2\mid \frac{m}{t}, r_3\mid \frac{m}{t} \\&\Leftrightarrow \lcm(r_1,r_2,r_3)\mid \frac{m}{t} \Leftrightarrow t \mid \frac{m}{ \lcm(r_1,r_2,r_3)},
\end{align*}
from which it follows that $ \gcd(n_1,n_2,n_3)=\frac{m}{ \lcm(r_1,r_2,r_3)}$.
\item[\enquote{(ii) $\Leftrightarrow$ (iii)}:]   Since $s_{12}=s_{13}=s_{23}=1$, any common multiple of $r_1,r_2,r_3$ is in fact a multiple of $r_1r_2r_3$, hence $\lcm(r_1,r_2,r_3)=r_1r_2r_3$.
\item[\enquote{(iii) $\Leftrightarrow$ (iv)}:] This follows straight from the definition $n_i=\frac{m}{r_i}$.
\item[\enquote{(iii) $\Leftrightarrow$ (v)}:] We have $\frac{n_1n_2n_3}{m}=\frac{m^2}{r_1r_2r_3}$, which equals $m$ iff $\frac{m}{r_1r_2r_3}=1$.
\item[\enquote{(iv) $\Rightarrow$ (vi)}:] For $\{i,j,l\}=\{1,2,3\}$, we have $\gcd(n_i,n_j)=\gcd(r_jr_l,r_ir_l)=r_ls_{ij}=r_l$.
\item[\enquote{(vi) $\Rightarrow$ (i)}:] Since $\gcd(n_1,n_2,n_3)$ must divide $\gcd(n_1,n_2)$, $\gcd(n_1,n_3)$, $\gcd(n_2,n_3)$ and these are pairwise coprime, it must be equal to $1$.
\end{DESCRIPTION}
\end{proof}

Thus $\frac{n_1n_2n_3}{m}=m=r_2n_2=\gcd(n_1,n_3)n_2$ by Lemma \ref{TFAE5} and using Lemma \ref{coprime}, we get
\begin{align*}
N&=a_4n_1n_2n_3-\frac{n_1n_2n_3}{m}-a_4(n_1n_2+n_1n_3+n_2n_3)-a_4-2+a_4(n_1+n_2+n_3)\\
&\hp+\gcd(n_1,n_2)+\gcd(n_1,n_3)+\gcd(n_2,n_3)\\
&=(a_4-1)(n_1-1)(n_2-1)(n_3-1)+(n_1-1)(n_2-1)(n_3-2)\\
&\hp+n_1n_2-(\gcd(n_1,n_3)+1)n_2-(n_1-\gcd(n_1,n_3)-1)+\gcd(n_2,n_3)+\gcd(n_1,n_2)-2\\
&=(a_4-1)(n_1-1)(n_2-1)(n_3-1)+(n_1-1)(n_2-1)(n_3-2)\\
&\hp+(n_2-1)(n_1-r_2-1)+r_1+r_3-2.
\end{align*}

We will define $B_5$ as the set of the following $N$ conjugates $\eta^{\sigma_1^{x_1}\sigma_2^{x_2}\sigma_3^{x_3}\sigma_4^{x_4}}$:
\begin{itemize}
\item $0\leq x_1<n_1-1$, $ 0\leq x_2<n_2-1$, $ 0\leq x_3<n_3-1$, $ 0<x_4\leq a_4-1$,
\item $0\leq x_1<n_1-1$, $ 0\leq x_2<n_2-1$, $ 1< x_3 \leq n_3-1$, $ x_4=0$,
\item $0\leq x_1< n_1-r_2-1$, $ 0\leq x_2<n_2-1$, $ x_3=0$, $ x_4=0$,
\item           $x_1=n_1-r_2-1$, $ 0\leq x_2<r_1+r_3-2$, $ x_3=0$, $ x_4=0$.
\end{itemize}
(Note that $n_3=r_1r_2>1$, $n_1-r_2-1=r_2(r_3-1)-1> 0$ and $r_1+r_3-2>0$, since $r_1,r_2,r_3>1$.)

\paragraph*{}
First we will recover the cases $0<x_4<a_4$, $x_1=n_1-1$ or $x_2=n_2-1$ or $x_3=n_3-1$ using the relations $N_1\sim 0$, $N_2\sim 0$, $N_3\sim 0$. From now on, we only need to deal with the cases where $x_4=0$. Next, we will recover the cases $1< x_3 \leq n_3-1$, $x_1=n_1-1$ or $x_2=n_2-1$ (and always $x_4=0$) using the relations $N_1\sim 0, N_2\sim 0$ and the cases $x_3=x_4=0$, $0\leq x_1< n_1-r_2-1$, $x_2=n_2-1$ using the relation $N_2\sim 0$.
\paragraph*{}
At this moment, we are only missing all the cases with $x_3=1$, $x_4=0$ and some of those with $x_3=x_4=0$. From now on, we will only focus on recovering those with $x_3=x_4=0$, because once we have those, we can recover those with $x_3=1,x_4=0$ just by using the relation $N_3\sim 0$.
\paragraph*{}
From now on, we will write $\overline{z}:=z\pmod{r_3}$ for any $z\in\Z$, so that $\overline{z}\in\{0,1,\dots,r_3-1\}$.
%$\uo:=r_2\pmod{r_3}$, $\vo:=r_1\pmod{r_3}$ % and similarly for other expressions
We will also define $h$ to be the unique integer satisfying $$r_1\cdot h\equiv r_2\pmod{r_3} \text{ and } h\in\{0,1,\dots,r_3-1\}$$ and similarly $h'$ to be the unique integer satisfying $$ r_2\cdot h'\equiv r_1\pmod{r_3} \text { and } h'\in\{0,1,\dots,r_3-1\}$$ (both are well defined, since $\gcd(r_1,r_3)=\gcd(r_2,r_3)=1$). Clearly $h\cdot h'\equiv 1 \pmod{r_3}$.

\paragraph*{}
Let $Q'$ be the quotient $\Z[G]$-module $$D^+/\big\langle \{\eta_I \big\vert \emptyset \subsetneq I \subsetneq P \}\big\rangle_{\Z[G]}$$
and let $Q$ be the quotient $\Z$-module of $Q'$ by the classes of conjugates we have already recovered, i.e.,
\begin{align*}
Q:=Q'/\big\langle \{\eta^{\sigma_1^{x_1}\sigma_2^{x_2}\sigma_3^{x_3}\sigma_4^{x_4}}; \quad & 0\leq x_1< n_1, 0\leq x_2<n_2, 0\leq x_3<n_3,0< x_4<a_4,\\
\text { or }& 0\leq x_1< n_1, 0\leq x_2<n_2, 1< x_3<n_3,x_4=0,\\
\text { or }& 0\leq x_1< n_1-r_2-1, 0\leq x_2<n_2, x_3=x_4=0,\\
\text { or }& x_1= n_1-r_2-1, 0\leq x_2<r_1+r_3-2, x_3=x_4=0 \}\big\rangle_{\Z}
\end{align*}
(where we denote $\eta^{\rho}\in D^+$ and its class in $Q'$ in the same way for any $\rho\in G$).
We will write $Q$ additively, denoting the class of $\eta$ in $Q$ by $\mu$, hence denoting the class of $\eta^{\rho}$ in $Q$ by $\rho\cdot \mu$ for any $\rho\in\Gal(k/\Q)$ or $\rho\in G$.
Showing that we have indeed chosen a basis now amounts to showing that $Q$ is trivial. Since $$0=\sigma_1^{x_1}\sigma_2^{x_2}N_3\cdot \mu=\sigma_1^{x_1}\sigma_2^{x_2}\cdot \mu+\sigma_1^{x_1}\sigma_2^{x_2}\sigma_3\cdot \mu$$
for any $x_1,x_2\in\Z$,
this is equivalent with showing that $\sigma_1^{x_1}\sigma_2^{x_2}\cdot \mu=0$ for each $0\leq x_1<n_1$, $0\leq x_2<n_2$.

\paragraph*{}
The conjugates with $x_3=0$ and $x_4=0$ (i.e., those of the form $\eta^{\sigma_1^{x_1}\sigma_2^{x_2}}$) can be visualized as a discrete rectangle with $n_1$ rows and $n_2$ columns. Since for each $x_4$, there are $n_3$ layers of such rectangles in total, the sum $\eta^{R_4\sum_{u=0}^{m-1}\sigma_1^{u}\sigma_2^{u}\sigma_3^{u}}$ must contain $\frac{m}{n_3}=r_3$ conjugates in each of these rectangles. We will now describe the sum of these. 

Let $$T:=\sum_{u=0}^{r_3-1}\sigma_1^{un_3}\sigma_2^{un_3}.$$

\begin{lemma}\label{Tdiag}
In $Q$, we have $$\sigma_1^{x_1}\sigma_2^{x_2}(1-\sigma_1\sigma_2)T\cdot \mu=0$$
for any $x_1,x_2\in\Z$.
\end{lemma}
\begin{proof}
Using the fact that every $0\leq w<m$ can be uniquely written as $un_3+v$, where $0\leq u<r_3$, $0\leq v<n_3$, together with the fact that the order of $\sigma_3$ is $n_3$, we get 
\begin{align*}
R_4T\sum_{v=0}^{n_3-1}\sigma_1^{v}\sigma_2^{v}\sigma_3^{v}=R_4\sum_{u=0}^{r_3-1}\sigma_1^{un_3}\sigma_2^{un_3}\sigma_3^{un_3}\cdot \sum_{v=0}^{n_3-1}\sigma_1^{v}\sigma_2^{v}\sigma_3^{v}=R_4\sum_{w=0}^{m-1}\sigma_1^{w}\sigma_2^{w}\sigma_3^{w}\sim 0.
\end{align*}
Together with $N_3\sim 0$, this means that 
\begin{align*}
0&\sim \sigma_1^{x_1}\sigma_2^{x_2}\left(R_4T \sum_{v=0}^{n_3-1}\sigma_1^{v}\sigma_2^{v}\sigma_3^{v}-\sigma_1\sigma_2N_3R_4T\right)=\sigma_1^{x_1}\sigma_2^{x_2}R_4T\sum_{v=0}^{n_3-1}\left(\sigma_1^{v}\sigma_2^{v}-\sigma_1\sigma_2\right)\sigma_3^{v}\\
&=\sigma_1^{x_1}\sigma_2^{x_2}(1-\sigma_1\sigma_2)R_4T+\sigma_1^{x_1}\sigma_2^{x_2}R_4T\sum_{v=2}^{n_3-1}\left(\sigma_1^{v}\sigma_2^{v}-\sigma_1\sigma_2\right)\sigma_3^{v}\\
&=\sigma_1^{x_1}\sigma_2^{x_2}(1\!-\!\sigma_1\sigma_2)T\!+\!
%\underbrace{
\sigma_1^{x_1}\sigma_2^{x_2}(1\!-\!\sigma_1\sigma_2)T\sum_{u=1}^{a_4-1}\sigma_4^u\!+\!\sigma_1^{x_1}\sigma_2^{x_2}R_4T\sum_{v=2}^{n_3-1}\left(\sigma_1^{v}\sigma_2^{v}-\sigma_1\sigma_2\right)\sigma_3^{v}.
%}_{
%\text{contains only terms with } x_4>0 \text{ or } x_3>1
%=0, \text{ as explained below}
%}
\end{align*}

Since all the summands in the expression $$\sigma_1^{x_1}\sigma_2^{x_2}(1-\sigma_1\sigma_2)T\sum_{u=1}^{a_4-1}\sigma_4^u+\sigma_1^{x_1}\sigma_2^{x_2}R_4T\sum_{v=2}^{n_3-1}\left(\sigma_1^{v}\sigma_2^{v}-\sigma_1\sigma_2\right)\sigma_3^{v}$$
have either $x_4>0$ or $x_3>1$ (where $x_3$ and $x_4$ denote the respective exponents of $\sigma_3$ and $\sigma_4$ in each term), the result of their action on $\mu$ becomes trivial in $Q$, which yields the result.
\end{proof}
\paragraph*{}
The rest of this section will again be stated purely algebraically, but perhaps it is helpful (although not strictly required) to see some of its parts geometrically.

We will decompose our rectangle (of conjugates of $\eta$ having $x_3=x_4=0$) into $r_3\times r_3$ rectangular blocks of height $r_2$ and width $r_1$ in the natural way. In the following, by a big row (resp. big column) we will understand a row of blocks (resp. columns), that is $r_3$ consecutive blocks next to (resp. above) each other. Since $r_2\mid n_3, r_1\mid n_3$ and the conjugates contained in $\eta^T$ are given by $\eta^{\sigma_1^{qn_3}\sigma_2^{qn_3}}$ for $0\leq q \leq r_3-1$, the Chinese remainder theorem implies that $\eta^{\sigma_1^{x_1}\sigma_2^{x_2}T}$ contains exactly one conjugate in every big row (resp. big column) for any $0\leq x_1< n_1, 0\leq x_2< n_2$, and these have the same relative position in each of the respective blocks (determined only by $\vo,\uo,x_1,x_2$). We can be even more precise: the horizontal distance between $\eta^{\sigma_1^{qn_3+x_1}\sigma_2^{qn_3+x_2}}$ and $\eta^ {\sigma_1^{(q+1)n_3+x_1}\sigma_2^{(q+1)n_3+x_2}}$ for $0\leq q \leq r_3-1$ and $0\leq x_1< n_1, 0\leq x_2< n_2$ is exactly $\uo\cdot r_1$, i.e., $\uo$ blocks, and the vertical distance between them is exactly $\vo\cdot r_2$, i.e., $\vo$ blocks (again this follows easily from the Chinese remainder theorem). It follows that the horizontal distance between any two conjugates in $\eta^T$ with a vertical distance of one block is $h$ blocks.

\paragraph*{}
For all $0\leq u\leq n_2$, we will denote $X_u:=\sigma_1^{n_1-2}\sigma_2^{u}\cdot \mu$ and $Y_u:=\sigma_1^{r_2(r_3-1)-1}\sigma_2^u\cdot \mu$. By definition, $X_u$ and $Y_u$ are elements of $Q$. It will be convenient to allow any integers in the indices of the $X$'s and $Y$'s and regard them only modulo $n_2$ (to be more precise, as in the set $\{0,1,\dots,n_2-1\}$). Moreover note that by definition, $Y_u=0$ for $0\leq u< r_1+r_3-2$. 

\begin{lemma}\label{XY}
We have $X_q=X_{q'}$ for any $q\equiv q'\pmod {r_3}$. Moreover,
for any $0\leq x_1<n_1$, $0\leq x_2<n_2$, we have
\begin{align*}
\sigma_1^{x_1}\sigma_2^{x_2}\cdot \mu=
\begin{cases}
0   &\text{ if }  x_1<r_2(r_3-1)-1,\\
Y_{x_2} \quad &\text{ if } x_1=r_2(r_3-1)-1, \\
X_{x_2-x_1-2}  \quad &\text{ if } r_2(r_3-1)\leq x_1<n_1-1, \\
X_{x_2-x_1-2}-Y_{x_2-h\cdot r_1} &\text{ if }  x_1=n_1-1.
\end{cases}
\end{align*}
\end{lemma}
\begin{proof}
The first part will be proven in a moment, we will now focus on the second.

The first case ($x_1<r_2(r_3-1)-1$) follows directly from the definition of $Q$ and the second case ($x_1=r_2(r_3-1)-1$) directly from the definition of $Y_{x_2}$.

\paragraph*{}
Now for every $0\leq u <n_2$, we will prove by induction with respect to $v=0,1,\dots,r_2-2$ that 
\begin{equation}\label{indX}
\sigma_1^{n_1-2-v}\sigma_2^{u-v}\cdot \mu =X_u.
\end{equation}

The base step $v=0$ is just the definition of $X_u$. Now suppose that $0<v\leq r_2-2$ and the statement holds for $v-1$. Then in the equality
\begin{equation}\label{Teq}
\left(\sigma_1^{n_1-2-v}\sigma_2^{u-v}(1-\sigma_1\sigma_2)\sum_{w=0}^{r_3-1}\sigma_1^{wn_3}\sigma_2^{wn_3}\right)\cdot \mu=0,
\end{equation}
which follows from Lemma \ref{Tdiag}, we claim that all the terms with $w>0$ do not contribute anything to the sum. Indeed, all the exponents of $\sigma_1$ are pairwise congruent modulo $r_2$ (since $r_2\mid n_3$), and since $n_1-r_2\leq n_1-2-v<n_1-2$ and $n_1-r_2+1\leq n_1-1-v<n_1-1$, we have $$\left(\sigma_1^{n_1-2-v}\sigma_2^{u-v}(1-\sigma_1\sigma_2)\sigma_1^{wn_3}\sigma_2^{wn_3}\right)\cdot \mu=0$$ for any $w>0$, because  $r_3$ does not divide $wn_3$ in this case. Hence \eqref{Teq} implies that
$$0=\left(\sigma_1^{n_1-2-v}\sigma_2^{u-v}(1-\sigma_1\sigma_2)\right)\cdot \mu=\sigma_1^{n_1-2-v}\sigma_2^{u-v}\cdot \mu-\underbrace{\sigma_1^{n_1-2-(v-1)}\sigma_2^{u-(v-1)}\cdot \mu}_{=X_u},$$
therefore $\sigma_1^{n_1-2-v}\sigma_2^{u-v}\cdot \mu=X_u$ by the induction hypothesis. This completes the induction, so \eqref{indX} holds.

\paragraph*{}
Now for any $0\leq u< n_2$, we will take $v=r_2-1$ in \eqref{Teq}. Again, since all the exponents of $\sigma_1$ are pairwise congruent modulo $r_2$ (since $r_2\mid n_3$) in this sum, the only terms which could be nonzero are those arising from $w=0$ and from $w$ satisfying
$$wn_3+n_1-2-(r_2-1)\equiv n_1-1 \pmod{n_1},$$
which is equivalent to $wn_3\equiv r_2\pmod{n_1}$, which implies $wn_3\equiv r_2\pmod{r_3}$. Together with $wn_3\equiv 0\pmod{r_1}$ and the fact that $\gcd(r_1,r_3)=1$, this means that the only solution to the above congruence is $wn_3\equiv h\cdot r_1\pmod{n_2}$.

Thus we have 
\begin{align*}
0&=\left(\sigma_1^{n_1-r_2-1}\sigma_2^{u-r_2+1}(1-\sigma_1\sigma_2)+\sigma_1^{n_1-1}\sigma_2^{u-r_2+1+h\cdot r_1}(1-\sigma_1\sigma_2)\right)\cdot \mu\\
&=\underbrace{\sigma_1^{n_1-r_2-1}\sigma_2^{u-r_2+1}\cdot\mu}_{=Y_{u-r_2+1}}-\underbrace{\sigma_1^{n_1-r_2}\sigma_2^{u-r_2+2}\cdot\mu}_{=X_u \text{ due to } \eqref{indX}}+\sigma_1^{n_1-1}\sigma_2^{u-r_2+1+h\cdot r_1}\cdot \mu\\
&\hp-\underbrace{\sigma_1^{n_1}\sigma_2^{u-r_2+1+h\cdot r_1+1}\cdot \mu}_{=0}.
\end{align*}

Therefore
\begin{equation}\label{indXY}
\sigma_1^{n_1-1}\sigma_2^{u-r_2+1+h\cdot r_1}\cdot \mu = X_u -Y_{u-r_2+1}.
\end{equation}

\paragraph*{}
Finally, for any $0\leq u< n_2$, we will take $v=r_2$ in \eqref{Teq}. Again, since all the exponents of $\sigma_1$ are pairwise congruent modulo $r_2$ in this sum, we only get nonzero terms for $w=0$ and for $w$ satisfying
$$wn_3+n_1-2-r_2\equiv n_1-2 \pmod{n_1},$$ which implies (because we have got the same congruence as above) $wn_3\equiv h\cdot r_1\pmod{n_2}$.

Thus we have 
\begin{align*}
0&\!=\underbrace{\sigma_1^{n_1-r_2-2}\sigma_2^{u-r_2}\cdot\!\mu}_{=0}\!-\!\underbrace{\sigma_1^{n_1-r_2-1}\sigma_2^{u-r_2+1}\cdot\!\mu}_{=Y_{u-r_2+1}}
\!+\!\underbrace{\sigma_1^{n_1-2}\sigma_2^{u-r_2+h\cdot r_1}\cdot \!\mu}_{=X_{u-r_2+h\cdot r_1}}\!-\!\underbrace{\sigma_1^{n_1-1}\sigma_2^{u-r_2+1+h\cdot r_1}\cdot \!\mu}_{=X_u-Y_{u-r_2+1} \text{ due to } \eqref{indXY}}.
\end{align*}

Therefore $X_{u -r_2+h\cdot r_1}=X_u$. Note that $$h\cdot r_1-r_2\equiv 0 \pmod{r_3}$$ and $$h\cdot r_1-r_2\equiv -r_2\pmod{r_1}.$$ Since $\gcd(-r_2,r_1)=1$ and $n_2=r_1r_3$, this means that for all $q,q'\in\Z$ satisfying $$q\equiv q'\pmod{r_3},$$ there is some $w\in\Z$ such that $$q'=w(h\cdot r_1-r_2)+q \pmod{n_2}.$$ 
Without loss of generality, we can assume that $w\ge 0$ (otherwise we can just swap $q$ and $q'$). But then $$X_{q}=X_{q+(h\cdot r_1-r_2)}=X_{q+2(h\cdot r_1-r_2)}=\dots=X_{q+w(h\cdot r_1-r_2)}=X_{q'}.$$

\paragraph*{}
Now for any $x_1,x_2$ satisfying $r_2 (r_3-1) \le x_1 < n_1 - 1$ and $0\le x_2 <x_2$, denoting $$v = n_1 - 2 - x_1,  u = v + x_2,$$
we get  $0 \le v \le r_2 - 2$ and the equality (1) implies
$$\sigma_1 ^{x_1} \sigma_2 ^{x_2} \mu = X_{n_1-2-x_1+x_2} 
= X_{x_2-x_1-2},$$
because $r_3 \mid n_1$.
%Now for any $0\leq u<n_2$, $0\leq v\leq r_2-2$ and $x_1=n_1-2-v,x_2=u-v$, the equality \eqref{indX} implies that $$\sigma_1^{x_1}\sigma_2^{x_2}\cdot \mu=X_u=X_{x_2-x_1-2},$$ since $$x_2-x_1-2=(u-v)-(n_1-2-v)-2=u-n_1\equiv u\pmod {r_3}.$$

Similarly, for $x_1=n_1-1$ and any $0 \le x_2 <n_2$, denoting $u=x_2+r_2-1-h\cdot r_1$, the equality \eqref{indXY} implies that 
$$\sigma_1^{x_1}\sigma_2^{x_2}\cdot \mu=X_u-Y_{u-r_2+1}=X_{x_2-x_1-2}-Y_{x_2-h\cdot r_1},$$ since 
$$u = x_2-1+r_2-h\cdot r_1 \equiv x_2-1\equiv x_2-2+1-n_1=x_2-x_1-2 \pmod{r_3}$$
by definition of $h$ and the fact that $r_3 \mid n_1$.

This concludes the proof.
\end{proof}

Thanks to Lemma \ref{XY}, from now on we will regard the indices of the $X$'s only modulo $r_3$. %(to be more precise, in the set $\{0,1,2,\dots,r_3-1\}$) and drop the bar notation for them. 
The lemma also implies the equality
\begin{equation}\label{Ycancel}
\sigma_1^{n_1-1}\sigma_2^{x_2}\cdot \mu+\sigma_1^{n_1-r_2-1}\sigma_2^{x_2-h\cdot r_1}\cdot \mu=X_{x_2-1}-Y_{x_2-h\cdot r_1}+Y_{x_2-h\cdot r_1}= X_{x_2-1}
\end{equation}
for any $x_2\in\Z$, which we will use several times. Another simple observation that will come in handy in the proofs of the following lemmas is that the unary operation of adding a fixed integer induces an automorphism of $\Z/\langle r_3\rangle$, which we will not mention explicitly anymore.

To show that $Q$ is trivial, it now suffices to show that $X_u=0$ for all $0\leq u< r_3$ and $Y_v=0$ for all $r_1+r_3-2\leq v< n_2$ (knowing already that  $Y_v = 0$ 
for all $0 \le v < r_1 +r_3 - 2$). To achieve this, we will use linear algebra.

%\includepdf[pages=-]{5_export.pdf}

Let 
\begin{equation*}
\alpha:= Y_{r_1+r_3-2}+Y_{r_1+r_3-1}+\dots+Y_{n_2-1}\in Q
\end{equation*}
and
\begin{equation}\label{beta}
\beta:=X_0+X_1+\dots+X_{r_3-1}\in Q.
\end{equation}

\begin{lemma}\label{AB}
We have $\alpha=\beta=0$.
\end{lemma}
\begin{proof}
Using the relation $N_2\sim 0$, we have $$0=\sigma_1^{r_2(r_3-1)-1}N_2\cdot \mu =\sum_{x_2=0}^{n_2-1}  \sigma_1^{r_2(r_3-1)-1}\sigma_2^{x_2}\cdot \mu= \sum_{x_2=0}^{n_2-1}Y_{x_2}=\alpha$$ and
\begin{align*}
0&= \sigma_1^{r_2(r_3-1)}N_2\cdot \mu=\sum_{x_2=0}^{n_2-1}  \sigma_1^{r_2(r_3-1)}\sigma_2^{x_2}\cdot \mu=\sum_{x_2=0}^{n_2-1}  X_{x_2-r_2(r_3-1)-2}\\
&=\sum_{x_2=0}^{r_1r_3-1}  X_{x_2+r_2-2}=
\sum_{u=0}^{r_1-1}\sum_{v=0}^{r_3-1} X_{ur_3+v+r_2-2}=r_1\cdot \sum_{v=0}^{r_3-1} X_{v+r_2-2}=r_1\cdot \beta,
\end{align*}
since each $x_2\in\{0,1,\dots,r_1r_3-1\}$ can be uniquely written as $ur_3+v$, where $0\leq u<r_1$, $0\leq v<r_3$.% and the unary operation of adding $r_2-2$ induces an automorphism of $\Z/r_3$.

Similarly, using Lemma \ref{XY} together with the relation $N_1\sim 0$ and the equality \eqref{Ycancel}, %(together with the fact that the operation of subtracting $h$ induces an automorphism of $\Z/r_3$),
we get
\begin{align*}
0&= \sum_{q=0}^{r_3-1} \sigma_2^{qr_1}N_1\cdot \mu =\sum_{q=0}^{r_3-1}\left(\sigma_1^{n_1-1}+\sigma_1^{r_2(r_3-1)-1}\right) \sigma_2^{qr_1}\cdot \mu   
+\sum_{x_1=r_2(r_3-1)}^{n_1-2}\sum_{q=0}^{r_3-1} \sigma_1^{x_1}\sigma_2^{qr_1}\cdot \mu\\
&=\sum_{q=0}^{r_3-1}(\sigma_1^{n_1-1}\sigma_2^{qr_1}+\sigma_1^{r_2(r_3-1)-1}\sigma_2^{(q-h)\cdot r_1})\cdot \mu
+\sum_{x_1=r_2(r_3-1)}^{n_1-2}\sum_{q=0}^{r_3-1} \sigma_1^{x_1}\sigma_2^{qr_1}\cdot \mu\\
&=\sum_{q=0}^{r_3-1}X_{qr_1-1}+\sum_{x_1=r_2(r_3-1)}^{n_1-2}\sum_{q=0}^{r_3-1}X_{qr_1-x_1-2}
= \sum_{x_1=r_2(r_3-1)}^{n_1-1}\sum_{q=0}^{r_3-1}X_{qr_1-x_1-2}
=r_2\cdot \beta,
\end{align*}
since for any $x_1$, all possible remainders modulo $r_3$ occur exactly once as the indices in the sum $\sum_{q=0}^{r_3-1}X_{qr_1-x_1-2}$ (due to the fact that the order of the class of $r_1$ is $r_3$ in $\Z/\langle r_3\rangle$, due to their coprimality).
Since $\gcd(r_1,r_2)=1$, this implies $\beta=0$ by Bézout's identity.
\end{proof}

Next, for $0\leq q \leq r_3-3$, we will define 
\begin{equation}\label{Gamma}
\Gamma_q:=\sum_{u=0}^{r_3-h'-1}\sum_{v=0}^{\uo-1}X_{q+v-ur_2-1}\in Q.
\end{equation}
\pagebreak
\begin{lemma}\label{G0}
For any $0\leq q \leq r_3-3$, we have $\Gamma_q=0$.
\end{lemma}
\begin{proof}
Using Lemma \ref{XY}, the relation $N_1\sim 0$ and the equality \eqref{Ycancel}, 
we get
\begin{align*}
0&= \sum_{u=0}^{r_3-h'-1} \sigma_2^{q-uhr_1}N_1\cdot \mu\\
&= \sum_{u=0}^{r_3-h'-2}  \underbrace{\left(\sigma_1^{n_1-1}\sigma_2^{q-uh r_1} +\sigma_1^{r_2(r_3-1)-1}\sigma_2^{q-(u+1)h r_1}\right)\cdot \mu}_{=X_{q-uhr_1-1} \text{ due to } \eqref{Ycancel}}\\
&\hp+\underbrace{\sigma_1^{r_2(r_3-1)-1}\sigma_2^q\cdot \mu}_{=Y_q}+
\underbrace{\sigma_1^{n_1-1}\sigma_2^{q-(r_3-h'-1)h r_1}\cdot\mu}_{=X_{q-(r_3-h'-1)hr_1-1}-Y_{q+r_1}}\\
&\hp+\sum_{x_1=r_2(r_3-1)}^{n_1-2}\sum_{u=0}^{r_3-h'-1} \sigma_1^{x_1}\sigma_2^{q-uh r_1}\cdot \mu.
\end{align*}
Now we will use the fact that $q\leq r_3-3\leq r_1+r_3-3$ (implying $Y_q=0$) and $$q-(r_3-h'-1)h r_1-hr_1=q-r_1r_3h+r_1hh'\equiv q+r_1\pmod{n_2},$$ since the congruence holds modulo both $r_1$ and $r_3$ (and $\gcd(r_1,r_3)=1$). Also note that $Y_{q+r_1}=0$, since
$$r_1\leq q+r_1\leq r_1+r_3-3,$$
which precisely justifies the bounds on $q$ that we used in the definition of $\Gamma_q$ and also explains why the upper bound in the first sum was chosen to be $r_3-h'-1$.

Continuing with the previous equality and using the congruence $hr_1\equiv r_2\pmod{r_3}$ and Lemma \ref{XY}, we thus have
\begin{align*}
0&= \left(\sum_{u=0}^{r_3-h'-2} X_{q-uhr_1-1}\right)+X_{q-(r_3-h'-1)hr_1-1}+\sum_{x_1=r_2(r_3-1)}^{n_1-2}\sum_{u=0}^{r_3-h'-1} X_{q-uh r_1-x_1-2}\\
&=\sum_{u=0}^{r_3-h'-1} X_{q-ur_2-1}+\sum_{x_1=r_2(r_3-1)}^{n_1-2}\sum_{u=0}^{r_3-h'-1} X_{q-ur_2-x_1-2}\\
&=\sum_{x_1=r_2(r_3-1)}^{n_1-1}\sum_{u=0}^{r_3-h'-1} X_{q-ur_2-x_1-2}.
\end{align*}

After using the substitution $v=n_1-1-x_1$, this becomes
\begin{align*}
0&=\sum_{u=0}^{r_3-h'-1} \sum_{v=0}^{r_2-1} X_{q+v-ur_2-1}\\
&=\sum_{u=0}^{r_3-h'-1} \left(\sum_{v=0}^{\overline{r_2}-1} X_{q+v-ur_2-1}+\sum_{v=\overline{r_2}}^{r_2-1} X_{q+v-ur_2-1}\right)\\
&=\sum_{u=0}^{r_3-h'-1} \sum_{v=0}^{\overline{r_2}-1} X_{q+v-ur_2-1}+\sum_{u=0}^{r_3-h'-1}\frac{r_2-\overline{r_2}}{r_3}\sum_{v=\overline{r_2}}^{\overline{r_2}+r_3-1} X_{q+v-ur_2-1}\\
&=\Gamma_q+\sum_{u=0}^{r_3-h'-1}\frac{r_2-\overline{r_2}}{r_3}\cdot \beta\\
&=\Gamma_q,
\end{align*}
since $\beta=0$ by Lemma \ref{AB}.
%since the unary operation of adding $q-ur_2-2$ induces an automorphism of $\Z/r_3$.
\end{proof}

Finally, let 
\begin{equation}\label{Delta}
\Delta:=\sum_{u=0}^{r_3-1}u\cdot\sum_{v=0}^{\uo-1} \sum _{w=0}^{\vo-1} X_{v+w-ur_2-1}\in Q.
\end{equation}

\begin{lemma}\label{D0}
We have $\Delta=0$.
\end{lemma}
\begin{proof}
Using Lemma \ref{XY}, the relation $N_1\sim 0$ and the equality \eqref{Ycancel},
we get
\begin{align*}
0&= \sum_{u=0}^{r_3-1}u\cdot \sum _{x_2=0}^{r_1-1}\sigma_2^{x_2-uh r_1}N_1\cdot \mu\\
&=\sum_{u=0}^{r_3-1}u\cdot \sum _{x_2=0}^{r_1-1} \left( \sigma_1^{n_1-1}\sigma_2^{x_2-uh r_1}+\sigma_1^{r_2(r_3-1)-1}\sigma_2^{x_2-uh r_1}\right)\cdot \mu\\
&\hp+\sum_{u=0}^{r_3-1}u\cdot \sum_{x_1=r_2(r_3-1)}^{n_1-2}\sum _{x_2=0}^{r_1-1}\sigma_1^{x_1}\sigma_2^{x_2-uh r_1}\cdot \mu\\
&=\sum_{u=0}^{r_3-2} \sum _{x_2=0}^{r_1-1} \left( u\cdot\underbrace{\sigma_1^{n_1-1}\sigma_2^{x_2-uh r_1}\cdot \mu}_{=X_{x_2-uhr_1-1}-Y_{x_2-(u+1)hr_1}}+(u+1)\cdot\underbrace{\sigma_1^{r_2(r_3-1)-1}\sigma_2^{x_2-(u+1)h r_1}\cdot \mu}_{=Y_{x_2-(u+1)h r_1}}\right)\\
&\hp+\sum_{x_2=0}^{r_1-1}(r_3-1)\cdot \underbrace{\sigma_1^{n_1-1}\sigma_2^{x_2-(r_3-1)h r_1}\cdot \mu}_{=
X_{x_2-(r_3-1)hr_1-1}-Y_{x_2-hr_1r_3}=
%X_{x_2-(r_3-1)hr_1-1}
}+\sum_{u=0}^{r_3-1}u\cdot \sum_{x_1=r_2(r_3-1)}^{n_1-2}\sum _{x_2=0}^{r_1-1}\sigma_1^{x_1}\sigma_2^{x_2-uh r_1}\cdot \mu,
\end{align*}

where we used the fact that $$x_2-hr_1r_3\equiv x_2\pmod{n_2}$$  and $0\leq x_2< r_1$, hence $Y_{x_2-hr_1r_3}=0$. Also note that for any $r_1\leq q<n_2$, there exist unique $$u\in\{0,1,\dots,r_3-2\},x_2\in\{0,1,\dots,r_1-1\}$$ such that $$q\equiv x_2-(u+1)hr_1 \pmod{n_2}$$ by the Chinese remainder theorem, since $\gcd(h,r_3)=1$ and for $u=r_3-1$, we would get  $q\equiv r\pmod{n_2}$, where $0\leq r<r_1$. Thus we get a bijection $$\{0,1,\dots,r_3-2\}\times\{0,1,\dots,r_1-1\}\to \{r_1,r_1+1,\dots,n_2-1\},$$ which we will use in a moment to transform a double sum into a simple one.

Continuing with the previous equality and using the congruence $hr_1\equiv r_2\pmod{r_3}$, we thus have

\begin{align*}
0&=\sum_{u=0}^{r_3-2} \sum _{x_2=0}^{r_1-1} u\cdot X_{x_2-ur_2-1}+\sum_{u=0}^{r_3-2} \sum _{x_2=0}^{r_1-1} Y_{x_2-(u+1)hr_1}+\underbrace{\sum_{q=0}^{r_1-1}Y_{q}}_{=0}\\
&\hp+\sum_{x_2=0}^{r_1-1}(r_3-1)\cdot X_{x_2-(r_3-1)r_2-1}+\sum_{u=0}^{r_3-1}u\cdot \sum_{x_1=r_2(r_3-1)}^{n_1-2}\sum _{x_2=0}^{r_1-1}X_{x_2-ur_2-x_1-2}\\
&=\sum_{u=0}^{r_3-1} \sum _{x_2=0}^{r_1-1} u\cdot X_{x_2-ur_2-1}+\underbrace{\sum_{q=r_1}^{n_2-1} Y_{q}+\sum_{q=0}^{r_1-1}Y_{q}}_{=\alpha}\\
&\hp+\sum_{u=0}^{r_3-1}u\cdot \sum_{x_1=r_2(r_3-1)}^{n_1-2}\sum _{x_2=0}^{r_1-1}X_{x_2-ur_2-x_1-2}\\
&=\alpha+\sum_{u=0}^{r_3-1}u\cdot \sum_{x_1=r_2(r_3-1)}^{n_1-1}\sum _{x_2=0}^{r_1-1}X_{x_2-ur_2-x_1-2}.
\end{align*}

After using the equality $\alpha=0$ by Lemma \ref{AB} and the substitutions $v=n_1-1-x_1$, %\nopagebreak[4] 
$w=x_2$, this becomes
\begin{align*}
%&=\sum_{u=0}^{r_3-1}u\cdot \sum_{x_1=r_2(r_3-1)}^{n_1-1}\sum _{x_2=0}^{r_1-1}X_{x_2-ur_2-x_1-2}\\
0&=\sum_{u=0}^{r_3-1}u\cdot \sum_{v=0}^{r_2-1}\sum _{w=0}^{r_1-1}X_{v+w-ur_2-1}\\
&=\sum_{u=0}^{r_3-1}u\cdot \sum_{v=0}^{r_2-1}\left(\sum _{w=0}^{\vo-1}X_{v+w-ur_2-1}+\sum _{w=\vo}^{r_1-1}X_{v+w-ur_2-1}\right)\\
&=\sum_{u=0}^{r_3-1}u\cdot \sum_{v=0}^{r_2-1}\sum _{w=0}^{\vo-1}X_{v+w-ur_2-1}+\sum_{u=0}^{r_3-1}u\cdot \sum_{v=0}^{r_2-1}\frac{r_1-\vo}{r_3}\cdot\sum _{w=\vo}^{\vo + r_3-1}X_{v+w-ur_2-1}\\
&=\sum_{u=0}^{r_3-1}u\cdot \sum _{w=0}^{\vo} \sum_{v=0}^{r_2-1} X_{v+w-ur_2-1}+\sum_{u=0}^{r_3-1}u\cdot \sum_{v=0}^{r_2-1}\frac{r_1-\vo}{r_3}\cdot \beta\\
&=\sum_{u=0}^{r_3-1}u\cdot \sum _{w=0}^{\vo} \sum_{v=0}^{r_1} X_{v+w-ur_2-1}\\
&=\sum_{u=0}^{r_3-1}u\cdot \sum _{w=0}^{\vo}\left( \sum_{v=0}^{\uo-1} X_{v+w-ur_2-1}+\sum_{v=\uo}^{r_2-1} X_{v+w-ur_2-1}\right)\\
&=\sum_{u=0}^{r_3-1}u\cdot \sum _{w=0}^{\vo} \sum_{v=0}^{\uo-1} X_{v+w-ur_2-1}+\sum_{u=0}^{r_3-1}u\cdot \sum _{w=0}^{\vo-1}\frac{r_2-\uo}{r_3}\cdot \sum_{v=\uo}^{\uo+r_3-1} X_{v+w-ur_2-1}\\
&=\Delta+\sum_{u=0}^{r_3-1}u\cdot \sum _{w=0}^{\vo-1}\frac{r_2-\uo}{r_3}\cdot \beta\\
&=\Delta,
\end{align*}
since $\beta=0$ by Lemma \ref{AB}.
\end{proof}

\paragraph*{}
Now let $\mathcal{X}$ be the free $\Z$-module with generators $\widehat{X}_0,\widehat{X}_1,\dots,\widehat{X}_{r_3-1}$.
Analogously to the definitions \eqref{beta}, \eqref{Gamma}, \eqref{Delta}, we will define 
\begin{align*}
\widehat{\beta}:&=\widehat{X}_0+\widehat{X}_1+\dots+\widehat{X}_{r_3-1}\in \mathcal{X},\\
\widehat{\Gamma}_q:&=\sum_{u=0}^{r_3-h'-1}\sum_{v=0}^{\uo-1}\widehat{X}_{\overline{q+v-ur_2-1}}\in \mathcal{X},\\
%&\text{ and }\\
\widehat{\Delta}:&=\sum_{u=0}^{r_3-1}u\cdot\sum_{v=0}^{\uo-1} \sum _{w=0}^{\vo-1} \widehat{X}_{\overline{v+w-ur_2-1}}\in \mathcal{X}
\end{align*}
for all $0\le q \le r_3-3$. Also let $\psi:\mathcal{X}\to Q$ be the $\Z$-module homomorphism satisfying $\psi(\widehat{X}_u)=X_u$ for all $0\le u <r_3$ (since $\mathcal{X}$ is free, this is well defined and determines $\psi$ uniquely). Then for all $0\le q \le r_3-3$, it's clear by Lemmas \ref{AB}, \ref{G0} and \ref{D0} that $$\psi(\widehat{\beta})=\beta=0, \psi(\widehat{\Gamma}_q)=\Gamma_q=0, \psi(\widehat{\Delta})=\Delta=0,$$
hence 
\begin{equation}\label{kerpsi}
\widehat{\beta}, \widehat{\Gamma}_q, \widehat{\Delta} \in \ker \psi.
\end{equation}
Since $\mathcal{X}$ is free, each of its elements can be expressed as $\sum_{c=0}^{r_3-1}c_u\widehat{X_u}$ for a unique $r_3$-tuple of integer coefficients $(c_0,c_1,\dots,c_{r_3-1})$. Using this correspondence, we will now construct a matrix $M$ with integer entries of size $r_3\times r_3$ (indexing its dimensions from $0$ to $r_3-1$) as follows:
%For any linear combination of the $X_u$'s, we can take the indices of all $X_u$ in the set $\{0,1,\dots,r_3-1\}$ (since $X_u=X_{\overline{u}}$ by Lemma \ref{XY}) and write the linear combination as $\sum_{c=0}^{r_3-1}c_uX_u$. By definition, $\beta,\Gamma_q$ and $\Delta$ are such linear combinations, and thus correspond to $r_3$-tuples of integer coefficients $(c_0,c_1,\dots,c_{r_3-1})$. Using this correspondence, we will now construct a matrix $M$ of size $r_3\times r_3$ (indexing its dimensions from $0$ to $r_3-1$) as follows:
\begin{itemize}
\item The $0$-th row will correspond to the coefficients of $\widehat{\beta}$ (i.e., it will consist of all 1's).
\item The $q$-th row for $1\leq q\leq r_3-2$ will correspond to the coefficients of $\widehat{\Gamma}_{q-1}$.
\item The $r_3-1$-th row will correspond to the coefficients of $\widehat{\Delta}$.
\end{itemize}

By the definition of $M$, we have 
\begin{equation}\label{Mdef}
M\cdot
\begin{pmatrix}
\widehat{X}_0\\ 
\widehat{X}_1 \\
\widehat{X}_2 \\ 
\widehat{X}_3 \\ 
\vdots\\ 
\widehat{X}_{r_3-2}\\
\widehat{X}_{r_3-1}
\end{pmatrix}
=
\begin{pmatrix}
\widehat{\beta}\\ 
\widehat{\Gamma}_0\\ 
\widehat{\Gamma}_1\\ 
\widehat{\Gamma}_2 \\ 
\vdots\\ 
\widehat{\Gamma}_{r_3-3} \\ 
\widehat{\Delta}
\end{pmatrix}
\end{equation}
% and $'$ denotes transposition. 
We need to show that $M$ is unimodular, i.e., invertible over $\Z$, from which it will follow that $\ker \psi=\mathcal{X}$, and consequently $X_u=0$ for all $0\le u <r_3$. To achieve that, we will study the effect of multiplying $M$ by a character matrix (i.e., basically performing the discrete Fourier transform). But first we will need two technical lemmas, which will prove useful in a while.
\paragraph*{}
Let
\begin{align*}
R(x):&=\sum_{q=0}^{r_3-1} x^q\in\Z[x],\\
D(x):&=\sum_{q=0}^{r_3-1}q\cdot x^q\in \Z[x],\\
P(x):&=-x^{r_2-1}\cdot \sum_{q=0}^{r_1-1} x^q\in \Z[x].
\end{align*}
%\paragraph*{}
%Let $$R(x):=\sum_{q=0}^{r_3-1} x^q\in\Z[x],$$
%$$D(x):=\sum_{q=0}^{r_3-1}q\cdot x^q\in \Z[x]$$
%and 
%$$P(x):=-x^{r_2-1}\cdot \sum_{q=0}^{r_1-1} x^q\in \Z[x].$$

\begin{lemma}\label{RD}
Let $\zeta\neq 1$ be any $r_3$-th root of unity. Then we have $R(\zeta)=0$ and $$D(\zeta)\cdot(\zeta-1)=r_3.$$
\end{lemma}
\begin{proof}
The first assertion is immediate since $R(\zeta)\cdot(\zeta-1)=\zeta^{r_3}-1=0$, but $\zeta\neq 1$. The second follows from the computation
\begin{equation*}
\begin{split}
D(\zeta)\cdot(\zeta-1)&=\sum_{q=1}^{r_3-1} q\cdot \zeta^{q+1}-\sum_{q=1}^{r_3-1} q\cdot \zeta^q=\sum_{q=2}^{r_3} (q-1)\cdot \zeta^{q}-\sum_{q=1}^{r_3-1} q\cdot \zeta^q\\
&=(r_3-1)\zeta^{r_3}+\sum_{q=1}^{r_3-1} (q-1)\cdot \zeta^{q}-\sum_{q=1}^{r_3-1} q\cdot \zeta^q\\
&=r_3-1-\sum_{q=1}^{r_3-1} \zeta^{q}\\
&=r_3- R(\zeta)\\
&=r_3.
\end{split}
\end{equation*}
\end{proof}

\begin{lemma}\label{argeo}
For any $b\in\Nbb$ and $y\in\Cbb$, we have the equality
$$(y-1)\cdot\sum_{u=1}^{b}u\cdot y^u=(b+1)y^{b+1}-\sum_{u=0}^b y^{u+1}.$$
\end{lemma}
\begin{proof}
We have 
\begin{align*}
(y-1)\cdot\sum_{u=1}^{b}u\cdot y^u&=\sum_{u=1}^{b}u\cdot y^{u+1}-\sum_{u=1}^{b}u\cdot y^u\\
&=\sum_{u=0}^{b}u\cdot y^{u+1}-\sum_{u=0}^{b-1}(u+1)\cdot y^{u+1}\\
&=b\cdot y^{b+1}+\sum_{u=0}^{b-1}(u-(u+1))\cdot y^{u+1}\\
&=b\cdot y^{b+1}+\underbrace{y^{b+1}-y^{b+1}}_{=0}+\sum_{u=0}^{b-1}-1\cdot y^{u+1}\\
&=(b+1)y^{b+1}-\sum_{u=0}^b y^{u+1}.
\end{align*}
\end{proof}

%Now let $\mathcal{X}$ be the free $\Z$-module with generators $\widehat{X_0},\widehat{X_1},\dots,\widehat{X_{r_3-1}}$. By abuse of notation, we can consider $\widehat{\beta},\widehat{\Gamma_q},\widehat{\Delta}\in \mathcal{X}$ for $0\leq q\leq r_3-3$, which formally look the same as $\beta,\Gamma_q,\Delta$ (except for the hats). Moreover 
Now let $\z$ be any $r_3$-th root of unity and consider the $\Z$-module homomorphism from $\mathcal{X}$ to the cyclotomic field $\Q(\z)$ given by $$\sum_{u=0}^{r_3-1}c_u \widehat{X}_u\mapsto \sum_{u=0}^{r_3-1}c_u \z^u$$ 
(since $\mathcal{X}$ is free, this is well defined and determines the homomorphism uniquely).
We can apply this homomorphism to $\widehat{\beta},\widehat{\Gamma_q},\widehat{\Delta}$ for any $0\le q \le r_3-3$, and we will denote its respective values on these elements by $\beta(\z),\Gamma_q(\z),\Delta(\z)\in\Q(\z)$. Note that since $\z^{r_3}=1$, we have $\z^u=\z^{\overline{u}}$ for any $u\in\Z$.
%these values depend on the indices of $X_u$ only modulo $r_3$, so it doesn't matter whether we regard these indices as in the set $\{0,1,\dots,r_3-1\}$ or just as integers. This will allow us to use the original definitions of $\beta,\Gamma_q,\Delta$ for their computation and switch between these two viewpoints liberally.

\begin{lemma}
Let $\zeta\neq 1$ be any $r_3$-th root of unity. Then for all $0\leq q<r_3-3$, we have $$\beta(\z)=0,$$ $$\Gamma_q(\zeta)=\zeta^{q}\cdot P(\zeta)$$ and $$\Delta(\zeta)=D(\zeta)\cdot P(\zeta).$$
\end{lemma}
\begin{proof}
Note that $\z^{-r_2}\neq 1$, since $\gcd(r_3,-r_2)=1$ and $\z\neq1$.

From the definitions and Lemma \ref{RD}, we directly get $\beta(\z)=R(\z)=0$. For the second assertion, we have
\begin{equation*}
\begin{split}
\Gamma_q(\zeta)&=\sum_{u=0}^{r_3-h'-1}\sum_{v=0}^{\uo-1}\z^{\overline{q+v-ur_2-1}}\\
&=\z^{q-1}\cdot\sum_{v=0}^{\uo-1}\z^{v}\sum_{u=0}^{r_3-h'-1}\z^{-ur_2}\\
&=\z^{q-1}\cdot(1+\zeta+\dots+\zeta^{\uo-1})(1+\zeta^{-r_2}+\zeta^{-2r_2}+\dots+\zeta^{-(r_3-h'-1)r_2})\\
&=\zeta^{q-1}\cdot\frac{\zeta^{\uo}-1}{\zeta-1}\cdot \frac{\zeta^{-(r_3-h')r_2}-1}{\zeta^{-r_2}-1}\\
&=\zeta^{q-1}\cdot\frac{\zeta^{r_2}-1}{\zeta^{-r_2}-1}\cdot \frac{\zeta^{r_1}-1}{\zeta-1}\\
&=-\zeta^{q}\cdot \zeta^{r_2-1}\cdot (1+\zeta+\zeta^2+\dots+\zeta^{r_1-1})\\
&=\zeta^{q}\cdot P(\zeta).
\end{split}
\end{equation*}

Similarly, using Lemma \ref{argeo} with $y=\z^{-r_2}$ and $b=r_3-1$, we can see that

\begin{equation*}
\begin{split}
\Delta(\zeta)&=
\sum_{u=0}^{r_3-1}u\cdot\sum_{v=0}^{\uo-1} \sum _{w=0}^{\vo-1} \z^{\overline{v+w-ur_2-1}}\\
&=\z^{-1}\cdot\sum_{v=0}^{\uo-1}\z^v \sum _{w=0}^{\vo-1} \z^{w}\sum_{u=0}^{r_3-1}u\cdot\z^{-ur_2}\\
&=\z^{-1}(1+\z+\!\dots\!+\z^{\uo-1})\\
&\cdot(1\!+\z+\dots+\z^{\vo-1})(\z^{-r_2}+2\z^{-2r_2}+\dots+(r_3-1)\z^{-(r_3-1)r_2})\\
&=\z^{-1} \cdot \frac{\zeta^{\uo}-1}{\zeta-1}\cdot \frac{\zeta^{\vo}-1}{\zeta-1}\cdot \frac{r_3\z^{-r_2r_3}-\sum_{u=0}^{r_3-1}\z^{-r_2(u+1)}}{\z^{-r_2}-1}
\\
&=\z^{-1} \cdot \frac{\zeta^{\uo}-1}{\zeta-1}\cdot \frac{\zeta^{\vo}-1}{\zeta-1}\cdot \frac{r_3(\z^{r_3})^{r_2}-\zeta^{-r_2}\cdot R(\zeta^{-r_2})}{\z^{-r_2}-1}
\\
&=\z^{-1}\cdot\frac{\zeta^{r_2}-1}{\zeta-1}\cdot \frac{\zeta^{r_1}-1}{\zeta-1} \cdot \frac{r_3}{\zeta^{-r_2}-1}\\
&=\z^{-1}\cdot\frac{r_3}{\zeta-1}\cdot \frac{\zeta^{r_2}-1}{\zeta^{-r_2}-1}\cdot \frac{\zeta^{r_1}-1}{\zeta-1}\\
&=-D(\zeta)\cdot \zeta^{r_2-1}\cdot (1+\zeta+\zeta^2+\dots+\zeta^{r_1-1})\\
&=D(\zeta)\cdot P(\zeta). \qedhere
\end{split}
\end{equation*}
\end{proof}

\begin{prop}\label{unimod}
$M$ is unimodular.
\end{prop}
\begin{proof}
Let $\zt$ be a primitive $r_3$-th root of unity and let $C$ be the corresponding $r_3\times r_3$ character matrix, i.e., $C=(\zt^{r\cdot c})_{0\leq r,c<r_3}$. We will use the two previous lemmas together with the fact that multiplying a column of successive powers of $\zt$ by a row of $M$ from the left corresponds to evaluating the polynomial obtained from this row at $\zt$. %After reindexing the dimensions of $M$ from $0$ to $r_3-1$, we have
Hence we have $M\cdot C=C'$, where $C'_{0,0}=R(1)=r_3$ and the $c-th$ column of $C'$ is
$$
\begin{pmatrix}
R(\zt^c)\\ 
P(\zt^c) \\ 
\zt^c \cdot P(\zt^c) \\ 
(\zt^{c})^2 \cdot P(\zt^c) \\ 
\vdots\\ 
(\zt^{c})^{r_3-3} \cdot P(\zt^c) \\ 
D(\zt^c) \cdot P(\zt^c)
\end{pmatrix}
=
\begin{pmatrix}
0\\ 
P(\zt^c) \\ 
\zt^c \cdot P(\zt^c) \\ 
\zt^{2c} \cdot P(\zt^c) \\ 
\vdots\\ 
\zt^{(r_3-3)c} \cdot P(\zt^c) \\ 
D(\zt^c) \cdot P(\zt^c)
\end{pmatrix}
$$
for any $0<c<r_3$ (we don't need to specify the rest of the $0$-th column, since it doesn't influence the determinant of $C'$). Thus by taking out $P(\zt^c)$ from each of these columns, we get (using that multiplication by $r_1$ is an automorphism of $\Z/\langle r_3\rangle$, since $\gcd(r_1,r_3)=1$)
\begin{align*}
|\det C'|&=|\det C''|\cdot\abs*{\prod_{0<c<r_3}P(\zt^c)}\\
&=|\det C''|\cdot\abs*{\prod_{0<c<r_3}-\zt^{c(r_2-1)}}\cdot\abs*{\prod_{0<c<r_3}\frac{\zt^{cr_1}-1}{\zt^c-1}}\\
&=|\det C''|,
\end{align*}

where
$$C''=
\begin{pmatrix}
r_3& 0& \dots & 0 & \dots & 0\\ 
*& 1& \dots & 1 & \dots & 1\\ 
*& \zt& \dots & \zt^c & \dots & \zt^{r_3-1}\\ 
*& \zt^2& \dots & \zt^{2c} & \dots & \zt^{2(r_3-1)}\\ 
\vdots& \vdots&\ddots  & \vdots & \ddots & \vdots\\ 
*& \zt^{r_3-3}& \dots & \zt^{(r_3-3)c} & \dots & \zt^{(r_3-3)(r_3-1)}\\ 
*& D(\zt)& \dots & D(\zt^c) & \dots & D(\zt^{r_3-1})\\ 
\end{pmatrix}.
$$
On the other hand, we can take the matrix $C$, add all of its rows to the $r_3-1$-th one (thus creating $\begin{pmatrix}
r_3 & 0 & 0& \dots &0
\end{pmatrix}$
there) and then, using the equality
$$-\zt^{(r_3-2)c}+\sum_{u=0}^{r_3-3}(u-r_3+1)\cdot\zt^{uc}=\sum_{u=0}^{r_3-1}u\cdot\zt^{uc}-(r_3-1)\cdot\underbrace{\sum_{u=0}^{r_3-1}\zt^{uc}}_{=0},$$ 
%we see that if we 
multiply the $(r_3-2)$-th row by $-1$ and add the $u$-th row multiplied by $(u-r_3+1)$ for each $0\le u \le r_3-3$,
so that the $r_3-2$-th row will become
$$\begin{pmatrix}
*& D(\zt)& \dots & D(\zt^c) & \dots & D(\zt^{r_3-1})
\end{pmatrix}.$$
Thus we will obtain a matrix with the same determinant as $C''$ (up to a sign). Since the elementary row operations preserve the determinant up to a sign, it follows that
$$|\det C|=|\det C''|=|\det C'|=|\det M|\cdot |\det C|.$$ 
Now, $C$ can be seen as a special type of a Vandermonde matrix, so we have $$\det C=\prod_{0\leq r<c<r_3}(\zt^r-\zt^c)\neq 0$$
(in fact it is well known that this equals $\pm \sqrt{r_3^{r_3}}$), which implies that %$|\det M'|=1$, hence%
 $|\det M|=1$, as needed.
\end{proof}

\begin{cor}
We have $X_u=0$ for all $0\leq q <r_3$.
\end{cor}
\begin{proof}
Let $M^{-1}$ be the inverse matrix to $M$. By Proposition \ref{unimod}, it exists and it has integer entries. From the equation \eqref{Mdef}, it then follows that
\begin{equation*}
\begin{pmatrix}
\widehat{X}_0\\ 
\widehat{X}_1 \\
\widehat{X}_2 \\ 
\widehat{X}_3 \\ 
\vdots\\ 
\widehat{X}_{r_3-2}\\
\widehat{X}_{r_3-1}
\end{pmatrix}
=
M^{-1}\cdot
\begin{pmatrix}
\widehat{\beta}\\ 
\widehat{\Gamma}_0\\ 
\widehat{\Gamma}_1\\ 
\widehat{\Gamma}_2 \\ 
\vdots\\ 
\widehat{\Gamma}_{r_3-3} \\ 
\widehat{\Delta}
\end{pmatrix},
\end{equation*}
which implies that $\widehat{\beta},\widehat{\Gamma}_0,\widehat{\Gamma}_1,\dots,\widehat{\Gamma}_{r_3-3}, \widehat{\Delta}$ generate $\mathcal{X}$. But all of these elements lie in $\ker \psi$ by \eqref{kerpsi}, hence $\ker \psi = \mathcal{X}$ and $\psi$ is the zero homomorphism. On the other hand, we know that the image of $\psi$ is generated by $X_0,X_1,\dots,X_{r_3-1}$ by the definition of $\psi$, so all of these must be zero as well.
\end{proof}

\begin{cor}
We have $Y_u=0$ for all $r_1+r_3-2\leq u < n_2$.
\end{cor}
\begin{proof}
By the Chinese remainder theorem, it suffices to show by induction with respect to $u=0,1,\dots,r_3-1$ that for any $0\leq v<r_1$, we have $Y_{v-uhr_1}=0.$ The base case $u=0$ follows directly from the definition of $Y_u$. Now suppose the statement is true for a given $0\leq u<r_3-1$. Then using $N_1\sim 0$ and Lemma \ref{XY}, we get
\begin{align*}
0&=\sigma_2^{v-uhr_1} N_1 \cdot \mu=\sum_{x_1=r_2(r_3-1)-1}^{n_1-1}\sigma_1^{x_1}\sigma_2^{v-uhr_1}\cdot\mu\\
&=\underbrace{Y_{v-uhr_1}}_{=0}-Y_{v-uhr_1-hr_1}+\sum_{x_1=r_2(r_3-1)}^{n_1-1}\underbrace {X_{v-uhr_1-x_1-2}}_{=0}=-Y_{v-(u+1)hr_1}
\end{align*}
by the induction hypothesis and the fact that $X=0$. This completes the induction.
\end{proof}
\pagebreak
By Lemma \ref{XY}, it now follows that $Q$ is trivial, so we have proven the following theorem:
\begin{theorem}\label{th5}
Under the assumptions on page \pageref{assum}, if $$a_1=a_2=a_3=r_4=1, r_1\neq 1, r_2\neq 1, r_3 \neq 1,s_{12}=s_{13}=s_{23}=1,\gcd(n_1,n_2,n_3)=1,$$ then  the set $B_{5}\cup B_D$ forms a basis of $D^+$ and the set $B_{5}\cup B_C$ forms a basis of $C^+$.
\end{theorem}
