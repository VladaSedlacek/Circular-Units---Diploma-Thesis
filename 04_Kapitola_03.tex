\chapter{Additional topics}

\section{The module of relations}
\section{Construction of suitable abelian fields}
Let $m,a_1,a_2,a_3,a_4,r_1,r_2,r_3,r_4$ be positive integers such that 
$$m>1, r_i\mid m, \gcd(r_i,r_j,r_l)=1, a_in_i\neq 1,$$
where $n_i=\frac{m}{r_i}$.
We will construct an infinite family of fields $k$ that satisfy all of our assumptions such that these integers correspond to the parameters in our problem of the same name.

First, we will fix distinct primes $p_1,p_2,p_3,p_4$ such that $p_i\equiv 1\pmod{ 2a_in_i}$ (by Dirichlet's theorem on primes in arithmetic progressions, there are infinitely many ways of doing this). Then there exist even Dirichlet characters $\chi_i$ of conductors $p_i$ and orders $a_in_i$ (namely, these can be given as $\chi_i:=\chi^{\frac{p_i-1}{a_in_i}}$, where $\chi$ is any generator of the cyclic group $\widehat{(\Z/p_i\Z)^\times}$ (note that $p_i>2$)).

Now let $K_i$ be the field associated to $\langle \chi_i\rangle$. Then $K_i$ is real (because $\chi_i$ is even) and $\Gal(K_i/\Q)$ is cyclic of order $a_in_i$, say $\Gal(K_i/\Q)=\langle \sigma_i\rangle$. Moreover, since the conductors $p_i$ are coprime, the group $\langle \chi_1,\chi_2,\chi_3,\chi_4\rangle$ corresponds to the compositum field $K=K_1K_2K_3K_4$. By the theory of Dirichlet characters, $K$ is ramified exactly at primes $p_i$ (with inertia subgroups isomorphic to $\Gal(K_i/\Q)$) and $$\Gal(K/\Q)=\Gal(K_1/\Q)\Gal(K_2/\Q)\Gal(K_3/\Q)\Gal(K_4/\Q)=\langle\sigma_1,\sigma_2,\sigma_3,\sigma_4\rangle,$$
so that $[K:\Q]=a_1a_2a_3a_4\frac{m^4}{r_1r_2r_3r_4}$. Now let $\tau:=\sigma_1^{a_1}\sigma_2^{a_2}\sigma_3^{a_3}\sigma_4^{a_4}$ and let $k$ be the subfield of $K$ fixed by $\tau$. Since $k$ is a subfield of a compositum of real fields, it must also be real. In order to reach our goal, we now only need to prove the following theorem (it is not hard to see that we could have used the results from Lemma \ref{comp} and Proposition \ref{degrees} as definitions instead).

\begin{theorem}
In the above notation, we have $[K:k]=m$, $[K:kK_i]=r_i$, $[k\cap K_i:\Q]=a_i$ and $kK_iK_jK_l=K$ (i.e. $K$ is the genus field in the narrow sense of $k$).
\end{theorem}
\begin{proof}
Using Lemma \ref{coprime} several times, we can compute
$$[K:k]=|\langle\tau\rangle|=\lcm\left(n_i,n_j,n_l\right)=m,$$
$$[K:kK_i]=|\langle\tau\rangle\cap \langle\sigma_j\sigma_l\sigma_h\rangle|=|\langle\tau^{a_in_i}\rangle|=r_i,$$
$$[k\cap K_i:/\Q]=[\langle\sigma_1,\sigma_2,\sigma_3,\sigma_4\rangle:\langle\tau,\sigma_j,\sigma_l,\sigma_h\rangle]=[\langle\sigma_1,\sigma_2,\sigma_3,\sigma_4\rangle:\langle\sigma_i^{a_i},\sigma_j,\sigma_l,\sigma_h\rangle]=a_i$$
and
$$[K:kK_iK_jK_l]=|\langle\tau\rangle\cap \langle\sigma_h\rangle|=|\langle\tau^{\lcm\left(n_i,n_j,n_l\right)}\rangle|=|\langle\tau^m\rangle|=1.$$
\end{proof}
