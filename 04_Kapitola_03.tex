\chapter{The module of relations}
In this chapter, we will try to study the relations between the generators of the group of circular numbers more abstractly, following the approach in \citep{Kucera2016}. Sometimes we will only state the results and omit the proofs, or just outline them.% at least outline some of the proofs.

Consider the (additively written) $\Z[G]$-module
\begin{align*}
\mathcal{X}:&=\bigoplus_{\emptyset\subsetneq I \subseteq \{1,2,3,4\}} \Z[\Gal(k\cap \prod_{i\in I}K_i)/\Q)]\\
&=\Z[\Gal(k/\Q)] \oplus \bigoplus_{i,j,l} \Z[\Gal(k\cap K_iK_jK_l)/\Q)]\\
&\hphantom{{}=} \oplus \bigoplus_{i,j} \Z[\Gal(k\cap K_iK_j)/\Q)]\oplus \bigoplus_{i} \Z[\Gal(k\cap K_i)/\Q)],
\end{align*}
where $G$ acts on each summand via restriction. For any $\emptyset\subsetneq I \subseteq \{1,2,3,4\}$, we will denote $x_I$ the element of $\mathcal{X}$ having all coordinates zero except for $1$ at the position corresponding to $I$. To simplify the notation, we will sometimes write simply $$x:=x_{\{1,2,3,4\}},x_{ijl}:=x_{\{i,j,l\}},x_{ij}:=x_{\{i,j\}},x_{i}:=x_{\{i\}}$$ and similarly $$\eta_{ijl}:=\eta_{\{i,j,l\}},\eta_{ij}:=\eta_{\{i,j\}},\eta_{i}:=\eta_{\{i\}}.$$
Therefore we have 
$$\mathcal{X}=\langle x,x_{123},x_{124},x_{134},x_{234}, x_{12},x_{13},x_{14},x_{23},x_{24},x_{34},x_{1},x_{2},x_{3},x_{4}\rangle_{\Z[G]}$$
and
$$D^+=\langle \eta,\eta_{123},\eta_{124},\eta_{134},\eta_{234}, \eta_{12},\eta_{13},\eta_{14},\eta_{23},\eta_{24},\eta_{34},\eta_{1},\eta_{2},\eta_{3},\eta_{4}\rangle_{\Z[G]}.$$

Since $$\eta\in k, \eta_{ijl}\in k\cap K_iK_jK_l, \eta_{ij}\in k\cap K_iK_j \text{ and } \eta_{i}\in k\cap K_i,$$ this gives us a surjective homomorphism of $\Z[G]$-modules $\varphi:\mathcal{X}\to D^+$ defined by $$\varphi(x)=\eta, \varphi(x_{ijl})=\eta_{ijl}, \varphi(x_{ij})=\eta_{ij}, \varphi(x_{i})=\eta_{i}.$$

Then $\ker\varphi$ is a $\Z[G]$-submodule of $\mathcal{X}$, and we will call it \textit{the module of relations}, because we can regard its elements as the relations between the generators of the group of circular numbers.

Lemmas \ref{units} and \ref{NoFrob} imply that for any $I \subseteq \{1,2,3,4\}$, $|I|\geq 2$ and $i\in I$, we have
$$\N_{k\cap \prod_{u\in I}K_u/ k \cap \prod_{u\in I\setminus{\{i\}}}K_u} \eta_I\in C^+\left( k \cap \prod_{u\in I\setminus{\{i\}}}K_u\right),$$
hence there exists some $$\rho_{i,I}\in \langle \{x_J | \emptyset\subsetneq J \subseteq I \setminus{\{i\}} \}\rangle_{\Z[G]}$$
 such that $$N_{i,I}:=\N_{k\cap \prod_{u\in I}K_u/ k \cap \prod_{u\in I\setminus{\{i\}}}K_u}x_I-\rho_{i,I} \in \ker \varphi.$$
 We will call $N_{i,I}$ a \textit{norm relation}.
 (Note that for $I=\{1,2,3,4\}$, we have $$\N_{k\cap \prod_{u\in I}K_u/ k \cap \prod_{u\in I\setminus{\{i\}}}K_u}x=R_iN_i x.)$$

\begin{rem}
In fact, the relation $N_{i,I}$ can be described much more explicitly using the Frobenius automorphisms, but we won't go into details here.
\end{rem}

Now let $M$ be the $\Z[G]$-submodule of $\ker \varphi$ generated by the norm relations $N_{i,I}$ for all possible $I \subseteq \{1,2,3,4\}$, $|I|\geq 2$ and $i\in I$. Our goal will be to describe the quotient $\Z[G]$-module $\ker\varphi/M$, which we will call \textit{the module of Ennola relations}. (However, to follow the terminology in \citep{Kucera2016}, by an \textit{Ennola relation} we will mean an element of $\ker\varphi\setminus M$ rather than $\ker\varphi/M$.)
%its elements \textit{Ennola relations}.

Let $E_{ijl}$ be the Ennola relation described by Theorem 10 in \citep{Kucera2016} applied to the field $k\cap K_iK_jK_l$. By Theorem 19 there, this is the unique Ennola relation (modulo the norm relations) for this field.
\begin{prop}\label{ennola}
In all the cases described in Chapter \ref{bases}, the $\Z[G]$-module $\ker\varphi/M$ is generated by the classes of $E_{123},E_{124},E_{134},E_{234}$ and the action of $G$ is trivial on it.
\end{prop}

\begin{proof}
For any case described in Chapter \ref{bases}, let $B$ be a $\Z$-basis of $D^+$. For any element of $B$, we will fix its preimage with respect to $\varphi$; let $Y$ be the set of these fixed preimages. Then the elements of $Y$ are $\Z$-linearly independent and we have $\mathcal{X}=\ker\varphi\oplus Y$. Recall that in order to construct $B$, we always used only ($\Z[G]$-linear combinations of) norm relations together with the four implicit Ennola relations $E_{123},E_{124},E_{134},E_{234}$ from \citep{Kucera2016}. This shows that $\ker \varphi$ is generated by $M\cup\{E_{123},E_{124},E_{134},E_{234}\}$, which proves the first part of the proposition. The second part follows from the observation that the action of $G$ on $E_{ijl}$ is the same as the action of $\Gal(k\cap K_iK_jK_l/\Q)$ on $E_{ijl}$, which is trivial by Theorem 19 in~\citep{Kucera2016}.
\end{proof}

\begin{rem}
In the case $a_1=a_2=a_3=a_4=r_1=r_2=r_3=r_4=1$ (which is a special case of the one in Section \ref{bases1}), it can be shown that $\ker \varphi/M \cong (\Z/m)^4$, which is a stronger result than in Proposition \ref{ennola}. The proof is too technical to be included here, but essentialy it consists of constructing a $\Z$-module (not $\Z[G]$-module!) homomorphism from $\mathcal{X}$ to $\Z/\langle m\rangle$ and showing that all the norm relations together with three of the four Ennola relations lie in its kernel, while the fourth Ennola relation maps to the class of $1$ modulo~$m$.
\end{rem}

\begin{rem}
A crucial part of the proof of Proposition \ref{ennola} was the fact that in all of the cases studied in Chapter \ref{bases}, we never encountered any new Ennola relation, i.e. an element of $\ker\varphi\setminus M$ having a nonzero coefficient at $x$. This will not always be case though, because we have already found a new Ennola relation $E$ in the special case $$m=a_3=r_3=2, a_1=a_2=a_4=r_1=r_2=r_4=1.$$ 
It's not very hard to show that $E\not\in M$ (and $2E\in M$), but the proof that $E\not\in\ker \varphi$ is again too technical to be described here). Note that in this case, we have $N=0$ (recall that $N$ was defined by the equation \eqref{Ndef}), but it is still possible to recover all the conjugates of $\eta$ using this new Ennola relation $E$.

In fact, it appears quite plausible that a new Ennola relation could arise whenever we have $a_i>1$ and $r_i>1$ at the same time. It is not a coincidence that this didn't happen in any of the cases studied in Chapter \ref{bases}, because it seems that this assumption will drastically increase the difficulty of the construction of $\Z$-bases of $D^+$ and $C^+$.
\end{rem}

\iffalse
\section{Construction of suitable abelian fields}
Let $m,a_1,a_2,a_3,a_4,r_1,r_2,r_3,r_4$ be positive integers such that 
$$m>1, r_i\mid m, \gcd(r_i,r_j,r_l)=1, a_in_i\neq 1,$$
where $n_i=\frac{m}{r_i}$.
We will construct an infinite family of fields $k$ that satisfy all of our assumptions such that these integers correspond to the parameters in our problem of the same name.

First, we will fix distinct primes $p_1,p_2,p_3,p_4$ such that $p_i\equiv 1\pmod{ 2a_in_i}$ (by Dirichlet's theorem on primes in arithmetic progressions, there are infinitely many ways of doing this). Then there exist even Dirichlet characters $\chi_i$ of conductors $p_i$ and orders $a_in_i$ (namely, these can be given as $\chi_i:=\chi^{\frac{p_i-1}{a_in_i}}$, where $\chi$ is any generator of the cyclic group $\widehat{(\Z/p_i\Z)^\times}$ (note that $p_i>2$)).

Now let $K_i$ be the field associated to $\langle \chi_i\rangle$. Then $K_i$ is real (because $\chi_i$ is even) and $\Gal(K_i/\Q)$ is cyclic of order $a_in_i$, say $\Gal(K_i/\Q)=\langle \sigma_i\rangle$. Moreover, since the conductors $p_i$ are coprime, the group $\langle \chi_1,\chi_2,\chi_3,\chi_4\rangle$ corresponds to the compositum field $K=K_1K_2K_3K_4$. By the theory of Dirichlet characters, $K$ is ramified exactly at primes $p_i$ (with inertia subgroups isomorphic to $\Gal(K_i/\Q)$) and $$\Gal(K/\Q)=\Gal(K_1/\Q)\Gal(K_2/\Q)\Gal(K_3/\Q)\Gal(K_4/\Q)=\langle\sigma_1,\sigma_2,\sigma_3,\sigma_4\rangle,$$
so that $[K:\Q]=a_1a_2a_3a_4\frac{m^4}{r_1r_2r_3r_4}$. Now let $\tau:=\sigma_1^{a_1}\sigma_2^{a_2}\sigma_3^{a_3}\sigma_4^{a_4}$ and let $k$ be the subfield of $K$ fixed by $\tau$. Since $k$ is a subfield of a compositum of real fields, it must also be real. In order to reach our goal, we now only need to prove the following theorem, thanks to Lemma \ref{genus}. %(it is not hard to see that we could have used the results from Lemma \ref{comp} and Proposition \ref{degrees} as definitions instead).

\begin{theorem}
In the above notation, we have $[K:k]=m$, $[K:kK_i]=r_i$, $[k\cap K_i:\Q]=a_i$ and $kK_iK_jK_l=K$.% (i.e. $K$ is the genus field in the narrow sense of $k$).
\end{theorem}
\begin{proof}
Using Lemma \ref{coprime} several times, we can compute
$$[K:k]=|\langle\tau\rangle|=\lcm\left(n_i,n_j,n_l\right)=m,$$
$$[K:kK_i]=|\langle\tau\rangle\cap \langle\sigma_j\sigma_l\sigma_h\rangle|=|\langle\tau^{a_in_i}\rangle|=r_i,$$
$$[k\cap K_i:/\Q]=[\langle\sigma_1,\sigma_2,\sigma_3,\sigma_4\rangle:\langle\tau,\sigma_j,\sigma_l,\sigma_h\rangle]=[\langle\sigma_1,\sigma_2,\sigma_3,\sigma_4\rangle:\langle\sigma_i^{a_i},\sigma_j,\sigma_l,\sigma_h\rangle]=a_i$$
and
$$[K:kK_iK_jK_l]=|\langle\tau\rangle\cap \langle\sigma_h\rangle|=|\langle\tau^{\lcm\left(n_i,n_j,n_l\right)}\rangle|=|\langle\tau^m\rangle|=1.$$
\end{proof}
\fi
