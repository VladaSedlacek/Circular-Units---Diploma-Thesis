\chapter*{Conclusion}
\addcontentsline{toc}{chapter}{Conclusion}

To summarize, we have managed to construct explicit $\Z$-bases of the groups of circular numbers and circular units (in Sinnott's sense) of a real abelian field with exactly four ramified primes in five different infinite families of cases. All of these are new results and they illustrate the power of the geometric approach, which is converted into algebraic notation afterwards. 
\paragraph*{}
However, even if there are some similarities in these five constructions, it seems that there is no easy way how to generalize all of them at the same time, so the general case remains open. It's quite probable that in order to solve it, we will first to fully understand new Ennola relations that will arise, unlike in our five cases. In this regard, it might also be useful to explore the relationship between all the Ennola relations, even those coming from the maximal subfields ramified at three primes. In the last chapter, we have briefly touched this subject, but the results here are much weaker than in Chapter \ref{bases}, so this seems to be a promising topic for future research.

%%%%%%%%%%%%%%%%%%%%%%%%%%%%%%%%%%%%
%%%%%%%%% GENERUJ TEXT %%%%%%%%%%%%%

%\shorthandoff{-}\lipsum[50-65]\shorthandon{-}
%%%%%%%%%%%%%%%%%%%%%%%%%%%%%%%%%%%%
