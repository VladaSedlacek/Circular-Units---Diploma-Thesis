\chapter*{Conclusion}
\addcontentsline{toc}{chapter}{Conclusion}

To summarize, we have managed to construct explicit $\Z$-bases of the groups of circular numbers and circular units (in Sinnott's sense) of a real abelian field with exactly four ramified primes in five different infinite families of cases (see Theorems \ref{th1}, \ref{th2}, \ref{th3}, \ref{th4} and \ref{th5}). All of these constructions  are new results and they illustrate the power of the geometric approach, which is converted into algebraic notation afterwards. 
\paragraph*{}
However, even if there are some similarities between these five constructions, it seems that there is no easy way how to generalize all of them at the same time, so the general case remains open. It's quite probable, that in order to solve it we will first need to fully understand the new Ennola relations that will arise, unlike in our five cases. In this regard it might also be useful to explore the relationship between all the Ennola relations, even those coming from the maximal subfields ramified at three primes. We have briefly touched this subject in the last chapter, but the results there are much weaker than in Chapter \ref{bases}, so this seems to be a promising topic for future research.

%%%%%%%%%%%%%%%%%%%%%%%%%%%%%%%%%%%%
%%%%%%%%% GENERUJ TEXT %%%%%%%%%%%%%

%\shorthandoff{-}\lipsum[50-65]\shorthandon{-}
%%%%%%%%%%%%%%%%%%%%%%%%%%%%%%%%%%%%
